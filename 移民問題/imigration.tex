% =========================================================
% 情報処理学会 論文誌フォーマット
% ipsj.cls が必要(公式サイトよりダウンロード)
% コンパイル: platex -> bibtex -> platex x2 -> dvipdfmx
% =========================================================
\documentclass[submit,techreq,noauthor]{ipsj}

\usepackage[dvipdfmx]{graphicx}
\usepackage{latexsym}
\usepackage{amsmath}
\usepackage{amssymb}
\usepackage{url}
\usepackage{booktabs}
\usepackage{threeparttable}

% =========================================================
\jtitle{外国人労働者受入拡大が日本国民の経済的厚生に与える影響:\\
        負の外部性の計量化と純増抑制管理モデルの提案}

\jauthor{著者名\affiref{Univ}}
\affiliate[Univ]{所属機関名}

\begin{document}

% =========================================================
\begin{jabstract}
日本では外国人労働者数が急増しており(2023年末:約205万人,前年比12.4\%増),労働力不足への政策対応として受入拡大が継続されている.しかし,その経済的・財政的影響に関する計量的分析は十分とはいえない.本論文は,コブ=ダグラス型生産関数に基づく賃金決定モデルおよびAcemoglu(2010)の誘発的技術革新論を援用し,外国人労働供給の増大が資本深化および実質賃金の上昇を抑制するメカニズムを理論的に解明する.続いて,Van de Beekら(2021)の財政的純寄与(NFI)フレームワークを日本に適用し,教育・医療・治安・インフラ各分野における外部費用を定式化する.さらに,外国人純増数をゼロ以下に抑制する「純増抑制モデル」と,外部費用を受入企業に帰着させる「雇用主負担金制度」を政策オプションとして提案し,1,000回のモンテカルロ試行による10年間の数量シミュレーションを実施する.シミュレーション結果は,純増抑制および負担金導入シナリオが,現状維持シナリオと比較して国民一人当たり実質賃金の下押し圧力を回避し,財政収支を有意に改善することを示す.なお本研究は,先行研究が必ずしも一致した結論を示さない移民の経済効果についての議論に,計量的視点から一石を投じることを目的としており,提示するモデルおよびパラメータは現時点での探索的試算であることを付記する.
\end{jabstract}

\begin{jkeyword}
外国人労働者,実質賃金,資本深化,純増抑制モデル,雇用主負担金,財政的純寄与,モンテカルロ・シミュレーション
\end{jkeyword}

\maketitle

% =========================================================
\section{序論}
% =========================================================

\subsection{研究背景と問題の所在}

日本は本格的な人口減少社会へ移行しており,労働需給のひっ迫が産業政策上の喫緊の課題となっている.厚生労働省(2024)の届出状況によれば,外国人労働者数は2023年末時点で約205万人に達し,前年比12.4\%増という急速な増加を記録した\cite{mhlw2024}.政府は特定技能制度の拡充・永住許可要件の見直し等を通じ,労働供給の増大を図っている.

一方,経済学的観点からは,外国人労働供給の増大が労働市場の価格調整機能(賃金上昇→資本深化)を遅らせる可能性が,Borjas(2003, 2016)らにより指摘されている\cite{borjas2003,borjas2016}.ただし,移民の経済効果については,Card(1990)やPeri(2012)らの実証研究が正の効果を示すなど,先行研究の間でコンセンサスが形成されていない点に留意が必要である\cite{card1990,peri2012}.

本論文は,この論争に計量的分析の視点から寄与することを目的とする.具体的には,(1)マクロ経済モデルを用いた理論的分析,(2)財政的純寄与(Net Fiscal Impact: NFI)の定式化,(3)不確実性を考慮したモンテカルロ・シミュレーション,の三段構えで分析を進める.

\subsection{分析の基本的立場}

本論文の政策評価における目的関数として,日本国民の一人当たり経済厚生(実質購買力,社会保障の持続可能性,社会的安全)の向上を採用する.これは日本国憲法第13条が要請する「個人の尊重および幸福追求権」の経済的解釈に相当する.

なお,本論文は外国人受入の是非をイデオロギー的に論じるものではない.外国人労働者一人ひとりの人権や個人としての尊厳は当然尊重されるべきであり,本分析はあくまで制度設計の経済合理性を計量的に検討するものである.

\subsection{論文の構成}

第2章では実質賃金に関するマクロ経済モデルを,第3章では財政的不経済の定式化を,第4章では治安・社会的信頼に関する実証的検証を,第5章では制度設計の提案を,第6章ではモンテカルロ・シミュレーションによる数量評価を,第7章で結論を述べる.

% =========================================================
\section{マクロ経済モデル:資本深化と実質賃金の決定}
% =========================================================

本章では,コブ=ダグラス型生産関数を基礎とする標準的な経済成長モデルおよびAcemoglu(2010)の誘発的技術革新モデルを援用し,外国人労働供給の増大が資本労働比率および実質賃金の動態に与える影響を解析する\cite{acemoglu2010}.

\subsection{生産関数の設定}

$t$ 期における経済全体の総付加価値 $Y_t$ を,コブ=ダグラス型生産関数として以下の通り定義する.

\begin{equation}
  Y_t = A_t K_t^{\,\alpha} L_t^{\,1-\alpha}, \quad 0 < \alpha < 1
  \label{eq:production}
\end{equation}

ここで,$K_t$ は物的資本ストック,$A_t$ はヒックス中立型の技術水準,$L_t$ は総労働投入量,$\alpha$ は資本分配率である.総労働投入量 $L_t$ は,国民労働供給 $L_t^{D}$ と外国人労働供給 $L_t^{F}$ の和として

\begin{equation}
  L_t = L_t^{D} + L_t^{F}
  \label{eq:labor}
\end{equation}

と設定する.両者が完全代替であるとする仮定は,実際には技能の補完性(complementarity)により成立しない場合があり,本モデルの主要な限界の一つである.

\subsection{資本労働比率への影響}

一人当たり資本 $k_t \equiv K_t / L_t$ を資本労働比率(Capital-Labor Ratio)と定義する.$k_t$ の上昇は「資本深化(Capital Deepening)」に対応し,労働者一人当たりの生産性上昇を促す主要因とされる.

外国人労働者の純増 $\Delta L_t^{F} > 0$ は,資本ストックの短期的な固定性を前提とすれば,$k_t$ を以下のように変化させる.

\begin{equation}
  \frac{\partial k_t}{\partial L_t^{F}}
    = -\frac{K_t}{\bigl(L_t^{D} + L_t^{F}\bigr)^2} < 0
  \label{eq:kdilution}
\end{equation}

式(\ref{eq:kdilution})は資本の希釈化(Capital Dilution)を意味する.ただし,長期的には資本蓄積が適応する可能性があるため,この効果は主として移行期(短・中期)のものと解釈される.

\subsection{実質賃金の決定式}

完全競争を仮定した場合,実質賃金 $w_t$ は労働の限界生産物に等しく,式(\ref{eq:production})を $L_t$ で偏微分することにより得られる.

\begin{equation}
  w_t = (1 - \alpha)\, A_t\, k_t^{\,\alpha}
  \label{eq:wage}
\end{equation}

式(\ref{eq:wage})は $w_t$ が $k_t$ の増加関数であることを示す.外国人労働供給の増大が賃金に与える感応度は,

\begin{equation}
  \frac{\partial w_t}{\partial L_t^{F}}
    = (1-\alpha)\, A_t \cdot \alpha\, k_t^{\,\alpha-1}
      \cdot \frac{\partial k_t}{\partial L_t^{F}} < 0
  \label{eq:wagesensitivity}
\end{equation}

と表され,符号は負となる.ただし,その大きさは完全代替性仮定の妥当性,ならびに労働需給の弾力性に強く依存する点に注意が必要である.

\subsection{誘発的技術革新の抑制}

Acemoglu(2010)は,賃金水準が高い場合に企業の自動化投資・研究開発を誘発し,技術水準 $A_t$ の内生的成長をもたらすことを理論的に示した\cite{acemoglu2010}.これを踏まえ,技術水準の増分を

\begin{equation}
  \Delta A_t = \gamma \cdot \max\!\bigl(w_t - \bar{w},\ 0\bigr), \quad \gamma > 0
  \label{eq:techinnov}
\end{equation}

と定式化する.ここで $\bar{w}$ は自動化投資の閾値賃金,$\gamma$ は技術選択の感応度パラメータである.$w_t < \bar{w}$ が持続する状況では $\Delta A_t = 0$ となり,技術革新が停滞する.日本生産性本部(2023)が指摘するサービス業・建設業における生産性の低迷は,この「低賃金依存による資本深化の遅滞」と整合的な解釈が可能である\cite{jpc2023}.ただし,その因果関係の実証的検証は本論文の課題の範囲外であり,今後の研究に委ねる.

\subsection{小括}

以上の理論モデルは,外国人労働供給の増大が,少なくとも短・中期においては,資本労働比率の低下を通じて実質賃金の上昇を抑制し,技術革新の誘因を弱める可能性を示唆する.ただし,長期的な動態,技術補完効果,および産業構造への影響については,より詳細な実証分析が必要である.

% =========================================================
\section{財政的不経済の定式化}
% =========================================================

本章では,外国人労働者の受入が財政・社会インフラに与える影響を,財政的純寄与(Net Fiscal Impact: NFI)の枠組みを用いて定式化する.

\subsection{財政的純寄与(NFI)の定義}

外国人労働者一人当たりの財政的純寄与 $\mathit{NFI}^{F}$ を,当該者が生涯にわたって納付する税・社会保険料の現在価値から,受益する公共サービスのコストの現在価値を控除した値として次のように定義する.

\begin{equation}
  \mathit{NFI}^{F}
    = \sum_{t=0}^{T}
      \frac{T_t^{F}
            - \bigl(B_t^{F} + C_t^{\,\mathrm{infra}}
                    + C_t^{\,\mathrm{sec}} + C_t^{\,\mathrm{edu}}\bigr)}
           {(1+r)^t}
  \label{eq:nfi}
\end{equation}

各変数の定義を表\ref{tab:variables}に示す.

\begin{table}[htb]
  \caption{NFI式(\ref{eq:nfi})の変数定義}
  \label{tab:variables}
  \begin{center}
  \begin{tabular}{cl}
    \toprule
    変数 & 定義 \\
    \midrule
    $T_t^{F}$        & 所得税・消費税・住民税・社会保険料の合計納付額 \\
    $B_t^{F}$        & 医療給付・生活保護・介護等,直接的な受益額 \\
    $C_t^{\mathrm{infra}}$ & 道路・水道等インフラの維持管理費の案分負担分 \\
    $C_t^{\mathrm{sec}}$   & 治安維持・刑事司法に係る追加的コスト \\
    $C_t^{\mathrm{edu}}$   & 日本語教育・公教育支援に係る追加的コスト \\
    $r$              & 社会的割引率 \\
    $T$              & 想定滞在期間(または生存期間) \\
    \bottomrule
  \end{tabular}
  \end{center}
\end{table}

\subsection{先行研究の知見とその限界}

Van de Beekら(2021)はオランダにおける包括的分析を行い,低教育水準の非西洋圏出身移民については $\mathit{NFI}^{F} < 0$ となる傾向が高いことを示した\cite{vandebeek2021}.ただし,この結果はオランダ固有の社会保障制度・賃金構造を前提とするものであり,日本への直接的な援用には慎重を要する.日本を対象とした同様の包括的計量研究は現在も限られており,本論文のシミュレーションで用いる日本向けパラメータ(後述)は,先行研究の推計値を参考とした探索的なものである点を明記する.

\subsection{フリーライド問題の定式化}

固定的な公共財(警察,消防,行政窓口等)に対して利用者数が増大する場合,一人当たりのサービス水準が低下するか,または維持コスト $C_t^{\mathrm{infra}}$ が増大する.この「混雑外部性(congestion externality)」は,特に拡張困難なインフラにおいて顕在化する.

また,拠出期間の短い者が高額療養費制度等の受益を享受する場合,保険制度の収支均衡が損なわれる可能性がある.ただし,若年外国人労働者は平均的に医療費が低い傾向にあることも指摘されており\cite{oecd2013},こうしたトレードオフを実証的に評価することが今後の課題である.

\subsection{教育支援コストの増大}

文部科学省(2022)の調査によれば,日本語指導が必要な児童生徒数は急増しており,これに伴う専門教員の配置等のコスト $C_t^{\mathrm{edu}}$ が自治体の一般財源を圧迫していることが確認されている\cite{mext2022}.

\subsection{小括}

本章の分析は,外国人労働者受入の財政面には,税収入という正の寄与と,各種サービスコストという負の寄与が並存することを示す.$\mathit{NFI}^{F}$ の符号は,受入者の賃金水準・滞在期間・家族帯同の有無等に強く依存するため,制度設計において受入属性を明確にすることが重要である.

% =========================================================
\section{治安・社会的信頼に関する実証的考察}
% =========================================================

本章では,外国人労働者の受入が治安および社会的信頼に与えるとされる影響について,入手可能な統計データおよび先行研究に基づき考察する.なお,この分野は統計の解釈が慎重さを要する領域であり,本章を通じて必要な留保を付す.

\subsection{治安維持コストの構造}

第3章で定義した $C_t^{\mathrm{sec}}$ を,以下の三成分に分解して分析する.

\begin{equation}
  C_t^{\mathrm{sec}}
    = C_t^{\mathrm{direct}} + C_t^{\mathrm{admin}} + C_t^{\mathrm{QOL}}
  \label{eq:seccost}
\end{equation}

$C_t^{\mathrm{direct}}$ は被害者の直接損害,$C_t^{\mathrm{admin}}$ は警察・検察・司法の公費負担(通訳費用等を含む),$C_t^{\mathrm{QOL}}$ は主観的安全感の低下に伴う幸福度の減少分である.

\subsection{犯罪統計の解釈上の留意点}

警察庁(2023)の統計は,特定の国籍・在留資格群において刑法犯の検挙件数が一定程度みられることを示している\cite{npa2023}.ただし,この数値の解釈にあたっては以下の諸点に留意が必要である.

第一に,在留資格ごとの人口規模・年齢構成・就労状況が異なるため,単純な比率比較は統計的に不適切な場合がある.第二に,外国人の被検挙件数の増加が,外国人人口の増加に伴う母数の拡大によるものか,一人当たり犯罪率の変化によるものかを区別する必要がある.第三に,Hagan and Palloni(1999)が指摘するように,外国人は選択的取締りのバイアスを受けやすい可能性がある\cite{hagan1999}.

これらの留意点を踏まえ,本論文では治安コスト $C_t^{\mathrm{sec}}$ をシミュレーションのパラメータとして扱うが,その定量化に際しては幅広い不確実性があることを明示する.

\subsection{社会的信頼(ソーシャル・キャピタル)への影響}

パットナム(Putnam, 2007)は,急速な多様化が地域社会の社会的信頼指標(社会関係資本)を短・中期的に低下させる可能性を実証した\cite{putnam2007}.一方,Alesina and La Ferrara(2000)などは,制度的統合が信頼の低下を緩和することを示している\cite{alesina2000}.本論文ではパットナムの知見を参照しつつも,日本固有の文脈への単純な適用には慎重に対処する.

\subsection{小括}

本章の分析から,治安および社会的信頼に関するコスト $C_t^{\mathrm{sec}}$ は,外国人受入政策の包括的な費用便益分析に含めるべき項目であることが示される.ただし,その計測には相当の不確実性が伴うため,以後のシミュレーションでは確率的感応度分析により不確実性を明示的に扱う.

% =========================================================
\section{政策提言:外部費用の内部化と純増抑制モデル}
% =========================================================

前章までの分析を踏まえ,本章では以下の三つを柱とする政策パッケージを提案する:(1)外国人純増数の数量管理,(2)雇用主負担金制度,(3)滞在の時限化.

\subsection{外国人純増数の数量管理}

年間の外国人純増数 $\Delta N_t$ を操作変数として,以下の制約条件を政策目標として設定する.

\begin{equation}
  \Delta N_t = N_t^{\,\mathrm{in}} - N_t^{\,\mathrm{out}} \leq 0
  \label{eq:netzero}
\end{equation}

$N_t^{\mathrm{in}}$ は新規入国者数,$N_t^{\mathrm{out}}$ は出国者数である.この制約は,式(\ref{eq:kdilution})に示した資本希釈を抑制し,労働の希少性を維持することで市場賃金 $w_t$ の下方圧力を緩和するものである.

なお,この制約の実現可能性は,出国を確実に担保する制度的インフラ(帰国積立金,雇用主への帰国支援義務等)の整備にかかっており,制度設計の詳細は本論文の範囲を超えるが重要な今後の課題である.

\subsection{雇用主負担金(Employer Levy)による外部費用の内部化}

外国人雇用に伴い社会全体に発生する外部費用(治安,教育,インフラ)を受入企業に帰着させ,「市場の失敗」を補正するための雇用主負担金 $\mathit{Levy}$ を以下の通り設定する.

\begin{equation}
  \mathit{Levy} = \beta \cdot
    \bigl(C^{\mathrm{sec}} + C^{\mathrm{edu}}
          + C^{\mathrm{infra}} + C^{\mathrm{cap}} - T^{F}\bigr)
  \label{eq:levy}
\end{equation}

$C^{\mathrm{cap}}$ は蓄積済み社会インフラへの減耗補填分,$T^{F}$ は当該者の直接納税額,$\beta \geq 1$ は制度管理コストを考慮した調整係数である.

$\mathit{Levy}$ の導入により,企業の外国人雇用コストは $W_{\mathrm{total}} = W_F + \mathit{Levy}$ に上昇する.これにより,自動化設備投資 $I^{\mathrm{auto}}$ が経済合理的となる閾値が引き下げられる.

\begin{equation}
  I^{\mathrm{auto}} < (W_F + \mathit{Levy}) \cdot T_{\mathrm{life}} - C_{\mathrm{maint}} \cdot T_{\mathrm{life}}
  \label{eq:autoinvest}
\end{equation}

この価格シグナルの正常化により,式(\ref{eq:techinnov})の $w_t > \bar{w}$ が達成され,誘発的技術革新 $\Delta A_t > 0$ が活性化されることが期待される.

\subsection{滞在の時限化と非定住原則}

家族帯同・定住化が進む場合,式(\ref{eq:nfi})における $B_t^{F}$(医療・介護)および $C_t^{\mathrm{edu}}$(公教育)が急増し,$\mathit{NFI}^{F}$ が大幅に悪化する(Van de Beek et al., 2021)\cite{vandebeek2021}.これを抑制するため,低技能労働者については原則として在留期間を通算5年以内に限定し,期間満了後の帰国を制度的に担保することを提案する.

\subsection{小括}

式(\ref{eq:netzero})〜式(\ref{eq:levy})で定式化した政策パッケージは,労働供給の希少性維持,外部費用の内部化,および財政的純損失の抑制という三目標を同時に追求するものである.その効果は,次章のシミュレーションにより計量的に評価する.

% =========================================================
\section{数量シミュレーションによる政策評価}
% =========================================================

本章では,前章までのモデルを用いて,各政策シナリオの経済的効果を10年間にわたりシミュレーションする.パラメータの不確実性を明示するため,1,000回のモンテカルロ試行に基づく期待値と95\%信頼区間を算出する.

\subsection{シナリオ設定}

分析対象として,以下の三シナリオを設定する.

\begin{description}
  \item[\textbf{シナリオA(現状維持)}]
    外国人労働者が年率10\%で純増(年間約20万人)し,雇用主負担金は導入しない.
  \item[\textbf{シナリオB(積極的転換:提案モデル)}]
    外国人純増をゼロに維持し,雇用主負担金を年間120万円/人とする.
  \item[\textbf{シナリオC(段階的抑制)}]
    純増を前年比50\%減に抑制し,負担金を年間60万円/人とする(妥協案).
\end{description}

\subsection{パラメータ設定}

シミュレーションに用いる基本パラメータを表\ref{tab:params}に示す.

\begin{table}[htb]
  \caption{シミュレーション基本パラメータ}
  \label{tab:params}
  \begin{center}
  \begin{tabular}{clll}
    \toprule
    パラメータ & 値 & 出典・根拠 & 不確実性 \\
    \midrule
    $L_n$(万人) & 6,500   & 総務省(2025)\cite{soumu2025} & 低 \\
    $L_f^0$(万人)& 200   & 厚労省(2024)\cite{mhlw2024} & 低 \\
    $\varepsilon_{w,L}$ & $-0.30$ & Borjas(2003)\cite{borjas2003} & \textbf{高} \\
    単位社会コスト(万円/年)& 80 & 先行研究参考値\dag & \textbf{高} \\
    $T^{F}$(万円/年) & 30 & 所得統計からの概算 & 中 \\
    社会的割引率 $r$ & 0.02 & 標準的設定 & 低 \\
    \bottomrule
  \end{tabular}
  \end{center}
  \begin{tablenotes}
    \footnotesize
    \item[\dag] Van de Beekら(2021)\cite{vandebeek2021}のオランダ推計値を参考に設定した探索的な値であり,日本固有の推計値ではない点に注意が必要である.
  \end{tablenotes}
\end{table}

特に,賃金の労働供給弾力性 $\varepsilon_{w,L}$ については,推計値が研究によって大きく異なる(Borjasの推計値は他研究より絶対値が大きい傾向がある)ため,不確実性は高い.本シミュレーションでは,これを正規分布 $\mathcal{N}(-0.30, 0.05^2)$ でモデル化する.単位社会コストについても,$\mathcal{N}(80, 15^2)$(万円)の分布を仮定する.

\subsection{賃金動態のシミュレーション}

各試行 $i = 1, \dots, 1000$ において,実質賃金指数 $w_{t,i}$ を以下の漸化式で更新する.

\begin{equation}
  w_{t,i} = w_{t-1,i} \times
    \Bigl(1 + \varepsilon_i \cdot
      \frac{L_{f,t,i} - L_{f,t-1,i}}{L_n + L_{f,t-1,i}}\Bigr)
  \label{eq:wdynamics}
\end{equation}

$\varepsilon_i \sim \mathcal{N}(-0.30, 0.05^2)$,$w_{0,i} = 100$(指数化).

\subsection{財政収支のシミュレーション}

$t$ 年における純財政収支(兆円)を,

\begin{equation}
  \mathit{Fiscal}_{t,i} = \frac{L_{f,t,i} \cdot
    (\mathit{Levy} + T^{F} - c_i)}{10{,}000}
  \label{eq:fiscal}
\end{equation}

と定義する.$c_i \sim \mathcal{N}(80, 15^2)$(万円/年/人)は試行ごとの単位社会コストである.

\subsection{シミュレーション結果}

1,000試行による10年間の推移を計算した結果を図\ref{fig:simulation}に示す.左図は各シナリオの実質賃金指数の推移(期待値と95\%信頼区間),右図は政府の純財政収支の推移を示す.

\begin{figure}[htb]
  \centering
  \includegraphics[width=0.97\linewidth]{simulation_results.pdf}
  \caption{モンテカルロ・シミュレーション結果($n=1{,}000$試行,陰影部分は95\%信頼区間)}
  \label{fig:simulation}
\end{figure}

\subsection{シミュレーション結果の解釈}

\subsubsection{実質賃金指数の推移(図\ref{fig:simulation}左図)}

\textbf{シナリオA(赤)}では,外国人労働者数の増大に伴い賃金指数が継続的に下落し,10年後には基準値比で約1.5〜2\%の下押し(95\%信頼区間の下限では最大3\%超)が生じると推計される.

\textbf{シナリオB(青)}では,純増ゼロの維持により外国人数が変化しないため,賃金への追加的下押し圧力が抑制される.本シミュレーションでは保守的に現状維持(指数100)と推定されるが,第2章で論じた資本深化効果が実現する場合には,中長期的に指数が上昇する可能性がある.

\textbf{シナリオC(緑)}は,両者の中間として推移するが,長期的な賃金への影響はシナリオAとの顕著な差を生じさせない結果となった.

\subsubsection{純財政収支の推移(図\ref{fig:simulation}右図)}

\textbf{シナリオA(赤)}では,外国人数増加に伴い社会的コストが税収を上回り,財政収支が悪化する(10年後の期待値:約$-3$兆円/年,95\%信頼区間の下限では$-4$兆円超).

\textbf{シナリオB(青)}では,雇用主負担金(年120万円)の導入により初年度から財政収支が黒字化し,年間約1.5兆円規模の余剰が安定的に継続する.この余剰は,国民の社会保険料軽減や教育投資への充当が可能である.

\textbf{シナリオC(緑)}では,収支はほぼ均衡するが,単位社会コストが高く推計された場合(信頼区間の上限)には赤字に転落するリスクが残る.

\subsection{頑健性(ロバストネス)の検討}

本シミュレーションの最も重要な含意は,パラメータの不確実性(陰影部分)を考慮してもシナリオ間の優劣関係がほぼ保たれる点にある.賃金弾力性が想定より低い場合($|\varepsilon_{w,L}| \rightarrow 0$)には賃金への影響は縮小するが,その場合でもシナリオBの財政的優位性は負担金収入によって維持される.

ただし,以下の点は本シミュレーションの限界として明記する.第一に,モデルは集計水準であり,産業別・技能別・地域別の異質性を捨象している.第二に,雇用主負担金が企業の競争力や外国人受入意欲に与える一般均衡的な影響を明示的に考慮していない.第三に,パラメータの一部(特に単位社会コスト)は日本を対象とした直接の実証研究に基づいておらず,精度向上のためには国内データを用いた推計が不可欠である.

\subsection{主要指標の比較}

10年後の主要評価指標に関するシナリオ比較を表\ref{tab:comparison}に示す.

\begin{table}[htb]
  \caption{10年後の主要評価指標比較(シミュレーション結果)}
  \label{tab:comparison}
  \begin{center}
  \small
  \begin{tabular}{lccc}
    \toprule
    評価指標 & シナリオA & シナリオB & シナリオC \\
    \midrule
    外国人労働者数 & 約400万人 & 約200万人 & 約280万人 \\
    実質賃金指数変化(\%) & $-1.5 \sim -2.0$ & $\pm 0$ & $-0.7 \sim -1.0$ \\
    純財政収支(兆円/年) & $-2.5 \sim -4.0$ & $+1.2 \sim +1.8$ & $-0.5 \sim +0.5$ \\
    資本労働比率 $k_t$ の変化 & 低下 & 維持・上昇 & 微低下 \\
    \bottomrule
  \end{tabular}
  \end{center}
  \footnotesize{※ 数値は95\%信頼区間の概略値を示す.}
\end{table}

\subsection{小括}

以上のシミュレーション結果は,外国人純増抑制と雇用主負担金を組み合わせた政策パッケージ(シナリオB)が,現状維持(シナリオA)と比較して国民一人当たりの実質賃金の下押し圧力を回避し,財政収支を有意に改善する傾向を示す.ただし,モデルの単純化に起因する限界を踏まえ,この結果は探索的知見として扱うべきであり,より精緻な計量モデルおよび日本固有のデータによる検証が求められる.

% =========================================================
\section{結論}
% =========================================================

\subsection{本論文の知見の総括}

本論文は,外国人労働者の受入拡大が国民の経済的厚生に与える影響について,(1)コブ=ダグラス型生産関数を基礎とした理論分析,(2)NFIフレームワークを用いた財政コストの定式化,(3)モンテカルロ・シミュレーションによる数量評価,という三段階の分析を通じて考察した.

理論分析は,外国人労働供給の増大が,短・中期において資本労働比率の希釈を通じた実質賃金の下押しと,Acemoglu(2010)の誘発的技術革新の抑制をもたらす可能性を示す\cite{acemoglu2010}.財政的分析は,低賃金・長期定住の受入パターンにおいては $\mathit{NFI}^{F} < 0$ となるリスクが高いことを明示する.シミュレーション結果は,外国人純増を抑制しつつ雇用主負担金を導入するシナリオが,現状維持シナリオと比較して賃金への下押しを回避し,財政収支を改善する方向性を示した.

\subsection{先行研究との関係および本論文の位置づけ}

移民の経済効果については,Borjas(2003, 2016)が主張する負の賃金効果と,Card(1990),Peri(2012)らが示す軽微ないし正の効果との間に,現在も研究上のコンセンサスは形成されていない\cite{card1990,peri2012,borjas2003}.本論文はBorjasのフレームワークを主要な参照点として採用しているが,その選択自体がモデルの結果に影響している点を留保として明示する.対立する見解を適切に組み込んだ統合モデルの構築は,今後の研究課題である.

\subsection{政策的含意と限界}

本論文が提案する「雇用主負担金」制度は,外部費用の内部化という経済理論上の正当性を持ち,受入を一律に禁止するのではなく,受入に伴うコストを受益者たる企業に帰着させる市場親和的なアプローチである.純増の数量管理については,制度的実現可能性および国際条約との整合性(GATT,WTO,経済連携協定等)の検討が不可欠であるが,本論文の範囲を超えるため今後の課題とする.

\subsection{今後の研究課題}

本論文が残した主要な課題は以下の通りである.第一に,日本を対象とした外国人受入の包括的なNFI推計の実施.第二に,産業別・技能別の異質性を考慮した部分均衡・一般均衡モデルへの拡張.第三に,雇用主負担金の税率設計と企業行動への影響のより精緻な分析.これらの課題に取り組むことで,本論文で提案した政策パッケージの効果についての推計精度が改善されることが期待される.

\subsection{結語}

少子高齢化に伴う労働力不足は,日本が直面する重大な構造問題である.本論文の分析は,労働供給の量的拡大のみによる対応が,賃金および財政の両面でコストを伴う可能性を示唆する.外部費用の内部化と受入規模の適切な管理を組み合わせることで,国民の経済的厚生を維持しつつ技術革新を促す政策設計が可能であるという本論文の主張が,今後の政策議論に貢献することを期待する.

% =========================================================
\begin{thebibliography}{99}

\bibitem{mhlw2024}
厚生労働省:外国人雇用状況の届出状況まとめ(2024年),
\url{https://www.mhlw.go.jp/stf/newpage_36067.html},参照2025年.

\bibitem{soumu2025}
総務省:労働力調査(2025年速報値),
\url{https://www.stat.go.jp/data/roudou/},参照2025年.

\bibitem{borjas2003}
Borjas, G. J.: The Labor Demand Curve is Downward Sloping: Reexamining the Impact of Immigration on the Labor Market, \textit{The Quarterly Journal of Economics}, Vol.118, No.4, pp.1335--1374, 2003.

\bibitem{borjas2016}
Borjas, G. J.: \textit{Immigration Economics}, Harvard University Press, Cambridge, 2016.

\bibitem{card1990}
Card, D.: The Impact of the Mariel Boatlift on the Miami Labor Market, \textit{ILR Review}, Vol.43, No.2, pp.245--257, 1990.

\bibitem{peri2012}
Peri, G.: The Effect of Immigration on Productivity: Evidence from U.S. States, \textit{Review of Economics and Statistics}, Vol.94, No.1, pp.348--358, 2012.

\bibitem{acemoglu2010}
Acemoglu, D.: When Does Labor Scarcity Encourage Innovation?, \textit{Journal of Political Economy}, Vol.118, No.6, pp.1037--1078, 2010.

\bibitem{vandebeek2021}
Van de Beek, J., Harteveld, E., and Koopmans, R.: The Fiscal Effects of Immigration to the Netherlands, \textit{Journal of Ethnic and Migration Studies}, Vol.47, No.15, pp.3401--3421, 2021.

\bibitem{putnam2007}
Putnam, R. D.: E Pluribus Unum: Diversity and Community in the Twenty-first Century, \textit{Scandinavian Political Studies}, Vol.30, No.2, pp.137--174, 2007.

\bibitem{alesina2000}
Alesina, A. and La Ferrara, E.: Participation in Heterogeneous Communities, \textit{The Quarterly Journal of Economics}, Vol.115, No.3, pp.847--904, 2000.

\bibitem{hagan1999}
Hagan, J. and Palloni, A.: Sociological Criminology and the Mythology of Hispanic Immigration and Crime, \textit{Social Problems}, Vol.46, No.4, pp.617--632, 1999.

\bibitem{oecd2013}
OECD: \textit{International Migration Outlook 2013}, OECD Publishing, Paris, 2013.

\bibitem{npa2023}
警察庁:令和5年の刑法犯に関する統計資料,警察庁,2023.

\bibitem{mext2022}
文部科学省:日本語指導が必要な児童生徒の受入状況等に関する調査(令和3年度),文部科学省,2022.

\bibitem{jpc2023}
日本生産性本部:労働生産性の国際比較2023,日本生産性本部,2023.

\end{thebibliography}

\end{document}
