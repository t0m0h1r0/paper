%% 投稿用 LaTeX 原稿(第4稿・第3回査読対応版)
%% タイトル:クォーラム型二重データセンタ構成における停止強度の漸近解析
%% ── 吸収型マルコフ連鎖の主固有値に基づく閉形式導出と数値検証 ──
%%
%% 改訂履歴:
%%   v1 : 初稿(要旨・第1〜6章・付録統合)
%%   v2 : 第1回査読対応(主要証明の補強・付録追加)
%%   v3 : 第2回査読対応(既約性命題・収束半径・CCF拡張・3DC比較など)
%%   v4 : 第3回査読対応(内部数学精査による修正)
%%        [M1] T_0 固有構造の誤り修正:
%%             「真の対角性・V_0=I・κ(V_0)=1」の主張を撤回.
%%             T_0 は下三角行列であり固有値の重複が起こり得るため一般に対角化不可.
%%             Kato 展開に必要な条件は「0 固有値が単純孤立」のみであることを明示し,
%%             それのみを証明する形に補題 3.4 を全面書き直し.
%%             Bauer--Fike(κ=1) 依存の箇所を注記で撤回・修正した旨を記載.
%%        [M2] 既約性証明の不完全箇所を修正:
%%             旧証明は「途中状態が吸収されない」ことの保証が欠落.
%%             正証明:Step1(修復は停止を引き起こさない),
%%             Step2(F(y) がいかなる最小カット集合も含まないから
%%             故障追加の中間状態はすべて T に属する)の2段階証明に置換.
%%        [M3] スペクトルギャップの表現修正:
%%             「γ ≥ μ_min(常時)」から「γ(ε) = μ_min + O(ε) ≥ μ_min/2」に修正.
%%             証明も Bauer--Fike(V_0=I) 依存をやめ,固有値の解析性+Lipschitz 連続性に変更.
%%        [M4] 各所の κ(V_0)=1 言及・寄与リスト・結論の記述を整合性修正.
%%
%% コンパイル: platex -> dvipdfmx  または  lualatex (luatexja)

\documentclass[submit,techreq]{ipsj}

\usepackage{amsmath,amssymb,amsthm}
\usepackage{graphicx}
\usepackage{booktabs}
\usepackage{array}
\usepackage{url}
\usepackage{algorithm}
\usepackage{algpseudocode}
\usepackage{longtable}

%% --- 定理環境 ---
\newtheorem{theorem}{定理}[section]
\newtheorem{proposition}[theorem]{命題}
\newtheorem{definition}[theorem]{定義}
\newtheorem{corollary}[theorem]{系}
\newtheorem{remark}[theorem]{注}
\newtheorem{lemma}[theorem]{補題}
\newtheorem{assumption}[theorem]{仮定}

%% --- 便利マクロ ---
\newcommand{\bE}{\mathbb{E}}
\newcommand{\calO}{\mathcal{O}}
\newcommand{\calA}{\mathcal{A}}   % 停止状態集合(Absorbing)
\newcommand{\calT}{\mathcal{T}}   % 過渡状態集合(Transient)
\newcommand{\calC}{\mathcal{C}}   % コンポーネント集合
\newcommand{\eps}{\varepsilon}
\newcommand{\e}{\mathrm{e}}
\newcommand{\RedRes}{S}            % 規約リゾルベント(reduced resolvent)
\newcommand{\Proj}{P}              % スペクトル射影

% ============================================================
\begin{document}
% ============================================================

\title{クォーラム型二重データセンタ構成における停止強度の漸近解析\\
{\Large ── 吸収型マルコフ連鎖の主固有値に基づく閉形式導出と数値検証 ──}}

\author{著者名 \and 著者名 \and 著者名}
\institute{所属機関名}
\date{\today}

\maketitle

% ============================================================
\begin{abstract}
本研究は,第三拠点にクォーラムノードを配置した二重データセンタ(2DC)
高可用性(HA)構成を対象とし,各構成要素の故障率および修復率が
システム停止強度に与える影響をスペクトル理論に基づいて定量評価するものである.
対象構成は二つのデータセンタ($A$, $B$),第三拠点クォーラムノード($Q$),
および三系統のネットワーク($N_{AB}$,$N_{AQ}$,$N_{BQ}$)から成る 6 要素系である.

各コンポーネントは相互独立な指数故障・修復過程に従うと仮定し,
吸収型連続時間マルコフ連鎖(CTMC)として定式化する($|\Omega|=64$状態,
過渡状態数 $|\calT|=36$,停止状態数 $|\calA|=28$).
停止強度(システム停止に至る指数的減衰率)$\theta$ を
過渡生成行列 $T$ の主固有値の絶対値として定義する.

業務継続条件の論理解析から最小停止カット集合がサイズ 2 の 5 集合
$\bigl\{\{A,B\},\{A,Q\},\{A,N_{BQ}\},\{B,Q\},\{B,N_{AQ}\}\bigr\}$
のみからなること(15 通りの 2 要素集合の完全列挙による)を証明し,
小故障率 $\eps_i = \lambda_i/\mu_i \to 0$ における摂動展開を通じて
\[
  \theta
  = \sum_{(i,j)\in\calC_2}
    \frac{\lambda_i\lambda_j}{\mu_i+\mu_j}
  + \calO(\eps^3),
  \quad \theta = \calO(\eps^2 \mu_{\min})
\]
の閉形式漸近展開を代数的に導出する($\calC_2$:最小停止カット集合の族).
一次項消失は $T_0$ の 0 固有値の単純性(下三角構造から証明)と生成行列の行和ゼロ性から厳密に証明し,
二次係数は規約リゾルベント展開の具体計算から閉形式で得る.
$T(\eps)$ の $\calT$ 上の既約性($F(y)$ が最小カット集合を含まない事実を用いた段階的証明)
から Perron--Frobenius 定理を適用し,主固有値の実数・単純性を保証する.
スペクトルギャップ $\gamma(\eps) = \mu_{\min} + \calO(\eps) \ge \mu_{\min}/2$ は固有値の解析的連続性から証明する.

数値固有値解析(64 状態,倍精度)および重要度サンプリング(IS)法
($N=5\times10^6$, 95\% CI 明示)により,閉形式近似の相対誤差 $<5\%$ および
$\eps^2$ スケーリング則(log-log 傾き $2.0\pm0.05$)を実証する.
本結果は,MTBF のみならず MTTR が停止確率に同次数で寄与することを
理論的に示し,クォーラム型 HA 設計の定量的指針を与える.
\end{abstract}

\begin{keywords}
連続時間マルコフ連鎖,吸収マルコフ過程,主固有値,Kato 摂動理論,
二重データセンタ,クォーラム型 HA 構成,最小カット集合,重要度サンプリング
\end{keywords}

% ============================================================
\section{序論}
\label{sec:intro}
% ============================================================

\subsection{研究背景}

ミッションクリティカルシステムにおける高可用性(HA)設計では,
二重化されたデータセンタ構成が標準的に採用されている
~\cite{avizienis2004,siewiorek1998,gray1993}.
特に,スプリットブレーン(脳分裂)防止のため第三拠点にクォーラムノードを配置する
2DC 構成は,ストレージシステム,分散データベース,
クラスタ管理機構等で広く採用されている~\cite{corbett2012,lamport1998,ongaro2014}.

従来の信頼性工学では,冗長構成の評価は信頼性ブロック図やカットセット理論に基づく
静的手法が中心であり~\cite{barlow1965,rausand2004,elsayed2012},
修復を伴う動的過程の厳密な解析は限られている~\cite{trivedi2002,shooman2002}.
修復可能システムに対する吸収型 CTMC の理論は
Keilson~\cite{keilson1979},Stewart~\cite{stewart1994},Norris~\cite{norris1997} らにより
体系化されており,主固有値と停止時間の指数的振る舞いは古典的に知られている.
固有値の解析的摂動については Kato~\cite{kato1995} が完全な理論的基盤を提供する.

一方,クォーラム理論は Gifford~\cite{gifford1979},Lynch~\cite{lynch1996},
Lamport~\cite{lamport1998} らの先駆的研究から発展し,Paxos,Raft~\cite{ongaro2014}
として実用化されている.クラウドインフラの可用性評価も活発に研究されている
~\cite{ford2010,dean2013,schroeder2007}が,
クォーラム型 2DC 構成を吸収型 CTMC として厳密定式化し,
故障率・修復率の双方を含む停止強度の閉形式を理論的に導出した研究は
著者らの知る限り存在しない.

\noindent
\textbf{Keilson(1979) との差別化について.}
修復可能系の指数的停止挙動と最小カット集合との関係は,
Keilson~\cite{keilson1979} が一般的枠組みとして確立している.
しかし本研究の固有の貢献は以下の 3 点にある:
(i)クォーラム整合性制約(スプリットブレーン防止論理)を持つ特定のネットワーク
     トポロジーの業務継続条件を,第一原理から形式的に定義し,
     $|\Omega|=64$ の全状態空間を完全分類($|\calA|=28$,$|\calT|=36$)した上で,
     最小停止カット集合を 15 通りの全 2 要素集合の網羅的検証により特定すること;
(ii)$T_0$ の下三角構造から 0 固有値の代数的重複度 1(単純孤立)を厳密に証明し,
     かつ $T(\eps)$ の $\calT$ 上の既約性($F(y)$ が最小カット集合を含まない事実を用いた段階的証明)から
     Perron--Frobenius 定理を適用して主固有値が実・単純・孤立であることを保証すること;
(iii)規約リゾルベントの具体的作用を代数的に追跡し,
     二次係数が最小停止カット集合の族 $\calC_2$ のみで決定されることを
     経路論的に解釈すること,および式~\eqref{eq:theta_main} の実用的閉形式を
     IS 法(95\% CI 付き)により独立に検証すること.
なお,収束半径 $\delta$ の陽的推定(補題~\ref{lem:analytic}),
時間スケール一様性の保証(定理~\ref{thm:approx} 注記),
および CCF の $\beta$-因子拡張(第~\ref{subsec:ccf}~節)は
Keilson(1979) では対象外の貢献である.

\subsection{本研究の貢献}

本研究の主な貢献は以下のとおりである:
\begin{enumerate}
  \item クォーラム型 2DC 構成(6 要素,64 状態)を吸収型 CTMC として厳密定式化し,
        停止状態集合($|\calA|=28$)・過渡状態集合($|\calT|=36$)を完全分類する
        (命題~\ref{prop:states},付録~\ref{app:cutsets}).
  \item 15 通りの 2 要素集合の完全列挙により,
        最小停止カット集合がサイズ 2 の 5 集合のみからなることを証明する
        (定理~\ref{thm:mincut},付録~\ref{app:cutsets}).
  \item $T(\eps)$ の $\calT$ 上の既約性を,$y \in \calT$ の故障集合 $F(y)$ が
        いかなる最小カット集合も含まないという事実に基づく段階的証明により確立し
        (命題~\ref{prop:irred}),Perron--Frobenius 定理により主固有値が実・単純・最大実部を持つことを保証する.
  \item $T_0$ の下三角構造から 0 固有値の単純性を証明し(補題~\ref{lem:T0spec},
        $T_0$ 全体の対角化可能性は不要),
        Kato 摂動理論~\cite{kato1995} を適用して一次項消失(補題~\ref{lem:a1zero})
        を証明する(収束半径 $\delta$ の陽的推定付き).
  \item 規約リゾルベント展開の具体計算により,
        停止強度の閉形式漸近展開を代数的に導出する
        (定理~\ref{thm:theta_main},$O(\eps^3)$ 定数のバウンド付き,付録~\ref{app:kato}).
  \item 固有値の解析性と連続性によりスペクトルギャップ $\gamma(\eps) = \mu_{\min} + \calO(\eps) \ge \mu_{\min}/2$ を証明し,
        指数近似の時間スケール一様性を保証する(補題~\ref{lem:specgap},定理~\ref{thm:approx}).
  \item 共通原因故障(CCF)の $\beta$-因子モデルによる拡張公式を導出し,
        独立故障仮定の適用限界を定量化する(第~\ref{subsec:ccf}~節).
  \item IS 法(95\% CI 付き)により $\eps^2$ スケーリング則を数値実証し,
        数値固有値解析との三者一致を確認する(第~\ref{sec:numerical}~章,付録~\ref{app:mc}).
\end{enumerate}

\subsection{論文構成}

第~\ref{sec:notation}~章では記号一覧,
第~\ref{sec:model}~章ではシステムモデル,
第~\ref{sec:analysis}~章では停止強度の漸近解析(主定理と完全証明,CCF 拡張を含む),
第~\ref{sec:numerical}~章では数値検証,
第~\ref{sec:design}~章では設計的示唆(3DC 比較を含む),
第~\ref{sec:conclusion}~章では結論を述べる.
付録~\ref{app:cutsets} に最小カット集合の完全証明,
付録~\ref{app:kato} に Kato 展開の完全代数計算,
付録~\ref{app:mc} に IS 法の擬似コードを示す.

% ============================================================
\section{記号・用語一覧}
\label{sec:notation}
% ============================================================

\begin{table}[h]
  \centering
  \caption{主要記号一覧}
  \label{tab:notation}
  \begin{tabular}{ll}
    \toprule
    記号 & 定義・単位 \\
    \midrule
    $\calC = \{A,B,Q,N_{AB},N_{AQ},N_{BQ}\}$ & コンポーネント集合(6 要素) \\
    $\lambda_i > 0$ & 要素 $i$ の故障強度($[\text{time}^{-1}]$) \\
    $\mu_i > 0$ & 要素 $i$ の修復強度($[\text{time}^{-1}]$) \\
    $\eps_i = \lambda_i/\mu_i$ & 無次元可用不全率 \\
    $\eps = \max_i \eps_i$ & 漸近展開の小パラメータ \\
    $\mu_{\min} = \min_i \mu_i$ & 最小修復強度($[\text{time}^{-1}]$) \\
    $\Omega = \{0,1\}^6$ & 状態空間($|\Omega|=64$) \\
    $\calA \subset \Omega$ & 停止状態集合($|\calA|=28$) \\
    $\calT = \Omega \setminus \calA$ & 過渡状態集合($|\calT|=36$) \\
    $Q \in \mathbb{R}^{64\times 64}$ & 全体生成行列 \\
    $T \in \mathbb{R}^{36\times 36}$ & 過渡部分行列 \\
    $\theta = -\nu_1 > 0$ & 停止強度($[\text{time}^{-1}]$):$T$ の主固有値の絶対値 \\
    $\calC_2$ & 最小停止カット集合(サイズ 2)の族 \\
    $\RedRes$ & 規約リゾルベント:$(T_0|_{\mathrm{Range}(I-\Proj_0)})^{-1}$ \\
    $\Proj_0 = \varphi_0 \psi_0^\top$ & 0 固有値に対する Riesz 射影($|\calT|=1$ の縮退なし) \\
    $\varphi_0,\, \psi_0$ & $T_0$ の 0 固有値に対応する右・左固有ベクトル \\
    $\tau$ & 停止到達時刻 \\
    $\gamma(\eps) \ge \mu_{\min}/2$ & スペクトルギャップ下界(補題~\ref{lem:specgap},$\gamma(\eps) = \mu_{\min} + \calO(\eps)$)\\
    \bottomrule
  \end{tabular}
\end{table}

% ============================================================
\section{システムモデルと停止強度の定式化}
\label{sec:model}
% ============================================================

\subsection{システム構成}

対象とする高可用性構成は以下の 6 コンポーネントから成る:
\begin{itemize}
  \item データセンタ本体:$A$,$B$(アクティブ–アクティブ運用,常時同期)
  \item 第三拠点クォーラムノード:$Q$
  \item 通信ネットワーク:$N_{AB}$(DC 間),$N_{AQ}$($A$–$Q$ 間),$N_{BQ}$($B$–$Q$ 間)
\end{itemize}

フェイルオーバー時間は無視し,切替は瞬時と仮定する
(非瞬時フェイルオーバーへの拡張は今後の課題とする).

\subsection{故障・修復モデル}

\begin{assumption}[独立指数分布]
\label{ass:indep}
各コンポーネント $i \in \calC$ は相互に独立な 2 状態過程に従い,
故障時間および修復時間はともに指数分布に従う(無記憶性).
故障強度 $\lambda_i > 0$,修復強度 $\mu_i > 0$,$\eps_i = \lambda_i/\mu_i \in (0,1)$.
\end{assumption}

\begin{assumption}[修復強度の下界]
\label{ass:mumin}
$\mu_{\min} := \min_{i\in\calC} \mu_i \ge c > 0$($c$ は $\eps$ に依存しない正定数).
\end{assumption}

\begin{assumption}[漸近設定]
\label{ass:asym}
スケーリング $\lambda_i = \eps \alpha_i$($\alpha_i > 0$,$\eps \to 0$)のもとで
主固有値の摂動展開を行う.すなわち $\eps_i = \eps \alpha_i/\mu_i$ である.
\end{assumption}

\subsection{状態空間と業務継続条件}

各コンポーネントの状態を $X_i(t) \in \{0,1\}$($1$:正常,$0$:故障)とし,
システム状態を $X(t) = (X_A, X_B, X_Q, X_{AB}, X_{AQ}, X_{BQ}) \in \Omega = \{0,1\}^6$
と定義する($|\Omega| = 64$).

\begin{definition}[業務継続条件]
\label{def:service}
業務継続可能条件 $\Phi(x) = 1$ を
\begin{equation}
  \Phi(x) = 1
  \iff
  (x_A \land x_B)
  \;\lor\;
  (x_A \land \lnot x_B \land x_Q \land x_{AQ})
  \;\lor\;
  (\lnot x_A \land x_B \land x_Q \land x_{BQ})
  \label{eq:service_cond}
\end{equation}
と定義する.停止状態集合 $\calA = \{x \in \Omega \mid \Phi(x) = 0\}$,
過渡状態集合 $\calT = \Omega \setminus \calA$.
\end{definition}

\begin{proposition}[状態数の明示]
\label{prop:states}
定義~\ref{def:service} のもとで $|\calA| = 28$,$|\calT| = 36$ が成立する.
\end{proposition}

\begin{proof}
停止条件 $\Phi(x) = 0$ を論理的に展開すると,2 類型に分類できる:

\noindent
\textbf{類型I(両 DC 停止):}
$x_A = 0 \land x_B = 0$.
残り 4 ビット $(x_Q, x_{AB}, x_{AQ}, x_{BQ})$ は任意値であるから
このカテゴリは $2^4 = 16$ 状態を含む.

\noindent
\textbf{類型II(スプリットブレーン停止):}
少なくとも一方の DC が正常($x_A \lor x_B = 1$)であるが,
DC 間ネットワーク断かつクォーラム不到達により合意不能になる停止,すなわち
$(x_A \lor x_B = 1) \land (\lnot x_A \lor \lnot x_B)$
かつ,稼働している DC からクォーラムへの経路が成立しない状態である.

式~\eqref{eq:service_cond} の否定の詳細展開(付録~\ref{app:cutsets})から,
$x_A = 0 \land x_B = 1$($A$ 停止,$B$ 正常)の停止条件は
$x_Q = 0 \lor x_{BQ} = 0$,
$x_A = 1 \land x_B = 0$($A$ 正常,$B$ 停止)の停止条件は
$x_Q = 0 \lor x_{AQ} = 0$
であるから,類型II の総数は $2 \cdot 3/4 \cdot 2^2 = 12$ 状態(重複なし)となる.

よって $|\calA| = 16 + 12 = 28$,$|\calT| = 64 - 28 = 36$.\qed
\end{proof}

\begin{remark}[$N_{AB}$ の役割とモデル仮定の範囲について]
\label{rem:NAB}
$N_{AB}$ は式~\eqref{eq:service_cond} に陽に現れない.
これは本モデルにおける \textbf{クォーラム到達可否} の定義が $N_{AQ}$・$N_{BQ}$ に依存し,
$N_{AB}$(DC 間直接リンク)は業務継続の必要条件ではないことを反映する:
アクティブ–アクティブ構成では,両 DC は $A$–$Q$,$B$–$Q$ 経路で独立にクォーラムへ到達可能であり,
$N_{AB}$ の故障はフェイルオーバー後の継続稼働を妨げない.

ただし,\textbf{データ同期(レプリケーション)の観点} では,
$N_{AB}$ の故障が同期遅延を引き起こし,フェイルオーバー時のデータ整合性に影響を与え得る.
本研究の業務継続条件式~\eqref{eq:service_cond} はクォーラム合意の成否(サービス提供可否)を記述するものであり,
レプリケーション遅延やダーティリードのリスクは対象外である
(これらは別途,可用性–整合性トレードオフ(CAP 定理~\cite{guerraoui2006,lynch1996})として扱われるべき問題である).
定理~\ref{thm:mincut} で示すように,
$N_{AB}$ を含むサイズ 2 の集合はいずれも最小停止カット集合ではない(表~\ref{tab:cutsets15}).
本モデルの適用範囲はクォーラム合意に基づくサービス稼働判断に限定されることを明示する.
\end{remark}

\subsection{吸収型 CTMC の生成行列}

仮定~\ref{ass:indep} のもとで,生成行列は
\begin{equation}
  Q = \begin{pmatrix} T & R \\ \mathbf{0}^\top & 0 \end{pmatrix},
  \quad T \in \mathbb{R}^{36\times36},
  \label{eq:Qpartition}
\end{equation}
と分解される~\cite{norris1997,keilson1979}.
$T$ の非対角要素は $q_{x,x^{(i)}} = \lambda_i$(故障遷移,$x^{(i)} \in \calT$)
または $q_{x,x^{(i)}} = \mu_i$(修復遷移)である.

\subsection{停止強度の定義}

\begin{definition}[停止時間と停止強度]
\label{def:theta}
停止時間を $\tau = \inf\{t > 0 \mid X(t) \in \calA\}$ と定義する.
生存確率ベクトル $p(t) = (P(X(t) = x,\, t < \tau))_{x \in \calT}$ は
$\dot{p}(t) = T\,p(t)$ に従う.
$T$ の固有値を $\nu_1 > \nu_2 \ge \cdots$(実部の降順)と並べると
$\Re(\nu_k) < 0$(すべての $k$)であり,
停止強度を
\[
  \theta := -\nu_1 = -\max_k \Re(\nu_k) > 0
  \quad (\,[\text{time}^{-1}]\,)
\]
と定義する.長時間では $|p(t)| \sim C\,\e^{-\theta t}$(定理~\ref{thm:approx}).
\end{definition}

\begin{remark}[$\theta$ と MTTF の関係]
\label{rem:mttf}
停止時間 $\tau$ が指数分布に従う場合(長時間近似が有効な領域),
平均停止時間(MTTF)は
\[
  \mathrm{MTTF} = \mathbb{E}[\tau] \approx \frac{1}{\theta}
  \quad (\text{指数近似の有効域 } t \gg \gamma^{-1} \text{ において})
\]
となる.ただし有限時間での $\mathbb{E}[\tau] = -\mathbf{1}^\top T^{-1} p_0$(正確式)は
指数近似とは異なる($T$ の全固有値の逆数の重み和).
本稿では停止強度 $\theta$ を主要指標とし,
MTTF との同一視は「$\theta t \ll 1$ かつ $\gamma t \gg 1$」の時間スケール
(混合時間より十分長く,停止イベントより十分短い中間域)に限定されることに注意する.
数値例では $\theta^{-1} \approx 119{,}000$ 年,$\gamma^{-1} \approx 1/365$ 年 $\approx 1$ 日であり,
実運用期間(数十年)ではこの条件は十分に成立する.
\end{remark}

% ============================================================
\section{停止強度の漸近解析}
\label{sec:analysis}
% ============================================================

\subsection{最小停止カット集合の完全証明}

\begin{definition}[停止カット集合と最小性]
\label{def:cutset}
$C \subseteq \calC$ が停止カット集合であるとは,
$\{x \in \Omega \mid x_i = 0,\, \forall i \in C\} \subseteq \calA$
が成立することをいう.
$C$ が最小停止カット集合であるとは,さらに $C$ のいかなる真部分集合も
停止カット集合でないことをいう.
\end{definition}

\begin{theorem}[最小停止カット集合の完全列挙]
\label{thm:mincut}
業務継続条件~\eqref{eq:service_cond} に対し,
最小停止カット集合の族は
\begin{equation}
  \calC_2 = \bigl\{
    \{A,B\},\; \{A,Q\},\; \{A,N_{BQ}\},\; \{B,Q\},\; \{B,N_{AQ}\}
  \bigr\}
  \label{eq:mincuts}
\end{equation}
であり,サイズ 1 の最小停止カット集合は存在せず,
$\calC_2$ はサイズ 2 の最小停止カット集合をすべて列挙する.
\end{theorem}

\begin{proof}
付録~\ref{app:cutsets} において以下を系統的に証明する:
(a)サイズ 1 の 6 集合はいずれも停止カット集合でない(各要素の単独故障に対する反例の構成),
(b)サイズ 2 の $\binom{6}{2} = 15$ 集合を全列挙し,
式~\eqref{eq:mincuts} の 5 集合が停止を保証すること,
残り 10 集合が業務継続可能な状態を持つこと(表~\ref{tab:cutsets15}),
を確認する.
最小性は(a)よりサイズ 1 の部分集合が停止カット集合でないことから保証される.\qed
\end{proof}

\begin{corollary}[最小カット集合サイズ]
\label{cor:kmin}
本構成の最小停止カット集合サイズは 2 である.
\end{corollary}

\subsection{$T(\eps)$ の既約性と Perron--Frobenius の適用}

\begin{proposition}[$\calT$ 上の既約性]
\label{prop:irred}
$\eps > 0$ のとき,過渡部分行列 $T(\eps)$ が生成する $\calT$ 上のマルコフ連鎖は既約である.
すなわち,任意の $x, y \in \calT$ に対して $x$ から $y$ へ $\calT$ 内のみを経由する正確率の有限パスが存在する.
\end{proposition}

\begin{proof}
任意の $x, y \in \calT$ を固定する.以下 2 段階でパスを構成する.

\noindent
\textbf{Step 1($x \to x_{\mathrm{up}}$:修復パスの安全性).}
$x$ のすべての故障コンポーネントを任意の順序で修復する有限遷移列を考える.
各修復ステップ $x \to x'$(1 ビット修復)では故障数が 1 減少し,
修復遷移は $\calA$ への遷移を引き起こさない:
$\calA$ は停止条件 $\Phi(x)=0$ を満たす状態の集合であり,
$x \in \calT$($\Phi(x)=1$,業務継続中)に修復を施した $x'$ は
$\Phi(x')=1$(故障が減ることで業務継続が保たれる).
よって修復ステップは常に $\calT$ 内に留まり,有限回で $x_{\mathrm{up}} \in \calT$ に到達する
($x_{\mathrm{up}}$ は $|\calT^{(0)}|=1$,補題~\ref{lem:T0spec}-(3) より $\calT$ に属する).

\noindent
\textbf{Step 2($x_{\mathrm{up}} \to y$:故障追加パスの安全性).}
$y$ の故障集合を $F(y) = \{i \mid y_i = 0\}$($|F(y)| = k \ge 0$)とおく.
$y \in \calT$ より,$F(y)$ は \textbf{いかなる最小停止カット集合も含まない}
($F(y)$ が最小カット集合 $C \in \calC_2$ を含めば $\{x \mid \forall i \in C,\, x_i=0\} \subseteq \calA$
 から $y \in \calA$ となり $y \in \calT$ に矛盾).

$F(y)$ の要素 $i_1, i_2, \ldots, i_k$ を任意の順序に並べ,故障を 1 つずつ追加する:
\[
  x_{\mathrm{up}} = z_0 \to z_1 \to \cdots \to z_k = y,
  \quad z_\ell = x_{\mathrm{up}} \text{ の } \{i_1,\ldots,i_\ell\} \text{ 番目ビットを 0 に変えた状態}.
\]
各中間状態 $z_\ell$ が $\calT$ に属することを確認する:
$z_\ell$ の故障集合は $\{i_1,\ldots,i_\ell\} \subseteq F(y)$ であり,
$F(y)$ のいかなる部分集合も最小停止カット集合(サイズ 2,定理~\ref{thm:mincut})を含まないから,
$z_\ell \notin \calA$($\Phi(z_\ell) = 1$).
最後に $z_k = y \in \calT$.

Step 1 と Step 2 を合成すると,$x \to x_{\mathrm{up}} \to y$ の有限パスが $\calT$ 内に存在する.
各遷移の速度は $\mu_i > 0$(修復)または $\eps\alpha_i > 0$(故障,$\eps>0$)であるから遷移確率は正.\qed
\end{proof}

\begin{remark}[Perron--Frobenius 定理の適用]
命題~\ref{prop:irred} より $-T(\eps)$($\eps>0$)は既約な M 行列であるから,
Perron--Frobenius 定理(Stewart~\cite{stewart1994}, 定理 1.4)により,
$T(\eps)$ の主固有値 $\nu_1(\eps)$ は \textbf{実数,代数的重複度 1},
かつ他の全固有値より大きい実部を持つことが保証される.
\end{remark}

仮定~\ref{ass:asym} のもとで $T(\eps) = T_0 + \eps T_1 + \calO(\eps^2)$ と書く:
\begin{itemize}
  \item $T_0$:$\eps = 0$($\lambda_i = 0$,修復のみ)での過渡生成行列.
        $T_0$ は非対角要素 $\mu_i$(修復遷移)のみを持つ.
  \item $T_1$:$\eps^1$ の係数行列(単一故障遷移項),
        非対角要素 $\alpha_i$($i$ 番目コンポーネントの故障遷移).
\end{itemize}

\begin{lemma}[$T_0$ の固有構造と 0 固有値の単純性]
\label{lem:T0spec}
\begin{enumerate}
  \item \textbf{(下三角構造)}
        故障数 $k = \sum_i(1-x_i)$ の昇順に $\calT$ の状態を並べると,
        $T_0$ は \textbf{下三角行列}(下三角部に非対角要素,対角は自己ループ)となる:
        $\eps=0$ では修復遷移 $\calT^{(k)}\to\calT^{(k-1)}$(故障数減少)のみ存在し,
        上三角はすべて 0.各 $k$-層内部には遷移なし(対角のみ).
        ただし \textbf{異なる状態間で固有値(対角値)が重複し得る}ため,
        $T_0$ は一般に対角化可能でないことに注意する(Jordan ブロックが生じ得る).

  \item \textbf{(固有値の集合)}
        三角行列の固有値は対角要素に等しい:
        \begin{equation}
          \sigma(T_0) = \Bigl\{-\!\!\!\sum_{i: x_i=0}\!\!\!\mu_i \;\Big|\; x \in \calT\Bigr\}
          \quad \text{(重複を込めた多重集合)}.
          \label{eq:T0eigs}
        \end{equation}

  \item \textbf{(0 固有値の単純性)}
        固有値 0 は状態 $x_{\mathrm{up}} = (1,1,1,1,1,1)$($k=0$)のみに対応し,
        \textbf{代数的重複度 1}(単純孤立固有値)である:
        $|\calT^{(0)}|=1$ より対角要素 0 は 1 回のみ出現し,
        $k\ge1$ の全状態で $\sum_{i:x_i=0}\mu_i \ge \mu_{\min} > 0$(固有値 $\le -\mu_{\min}$).
        よって 0 の孤立ギャップは $\mu_{\min}$.

  \item \textbf{(右・左固有ベクトル)}
        0 固有値に対応する右固有ベクトル $\varphi_0$ と左固有ベクトル $\psi_0$:
        \begin{align}
          T_0\, \mathbf{e}_{x_{\mathrm{up}}} &= \mathbf{0}
            \;\Rightarrow\; \varphi_0 = \mathbf{e}_{x_{\mathrm{up}}},
            \label{eq:rv0}\\
          \mathbf{1}_{36}^\top T_0 &= \mathbf{0}^\top
            \;\Rightarrow\; \psi_0 = \mathbf{1}_{36},
          \label{eq:lv0}
        \end{align}
        規格化条件 $\psi_0^\top\varphi_0 = 1$ を満たす.
\end{enumerate}
\end{lemma}

\begin{proof}
\textbf{(1)(2)} $\eps=0$ では故障遷移は存在せず,非対角要素 $(T_0)_{xy}$($x\ne y$)は
修復遷移 $q_{xy}=\mu_i$($y$ は $x$ の $i$ 番目ビットを 1 に変えた状態,$x_i=0$)のみ.
$|y| = |x|-1 < |x|$ であるから,故障数昇順の行列表現で $T_0$ は真の下三角行列(上三角 0).
三角行列の固有値は対角要素に等しい.

\textbf{(3)} $|\calT^{(0)}|=1$($x_{\mathrm{up}}$ のみ)より対角要素 0 は 1 回のみ現れ,
代数的重複度 1.$k\ge1$ 全状態で固有値 $\le -\mu_{\min}$,孤立ギャップ $\mu_{\min}$.

\textbf{(4)} $x_{\mathrm{up}}$ からの修復遷移は存在しない(故障コンポーネントなし)かつ
$(T_0)_{x_{\mathrm{up}},x_{\mathrm{up}}}=0$ より $T_0\mathbf{e}_{x_{\mathrm{up}}}=\mathbf{0}$.
$\mathbf{1}^\top T_0=\mathbf{0}^\top$ は $\eps=0$ での行和ゼロ性
($T_0$ の全行和ゼロ:吸収は 2 重以上の故障が必要で $\eps=0$ では起こらない)から従う.\qed
\end{proof}

\begin{remark}[$T_0$ の Jordan 構造と Kato 展開の適用]
\label{rem:jordan}
$T_0$ は下三角だが固有値が重複し得るため,一般に \textbf{対角化不可能}(欠陥あり)の可能性がある.
しかし Kato 展開が必要とする条件は \textbf{0 固有値が単純孤立固有値であること} のみ
(Kato~\cite{kato1995}, 定理 II-1.8)であり,$T_0$ の Jordan 構造の詳細は不要である.
0 固有値に対するRiesz 射影 $\Proj_0 = \varphi_0\psi_0^\top$(補題~\ref{lem:T0spec}-(4))は
右・左固有ベクトルのみで決定され,規約リゾルベント
$\RedRes = (-T_0|_{\mathrm{Ran}(I-\Proj_0)})^{-1}$ も well-defined である
($T_0$ の 0 以外の固有値はすべて $\le -\mu_{\min}$,補完空間上で $T_0$ は可逆).
\textbf{Bauer--Fike 定理の適用}については,後述のスペクトルギャップ補題~\ref{lem:specgap} において
$T_0$ を対角化せずに扱う別の証明に切り替える.
\end{remark}

この補題より,0 固有値の右・左固有ベクトルは
\begin{equation}
  \varphi_0 = \mathbf{e}_{x_{\mathrm{up}}}, \quad
  \psi_0 = \mathbf{1}_{36},
  \quad \psi_0^\top \varphi_0 = 1
  \label{eq:eigvec}
\end{equation}
と規格化でき,Kato 展開の適用に必要な条件がすべて満たされる(注記~\ref{rem:jordan}).

\subsection{Kato 摂動理論の適用}

\begin{lemma}[固有値の解析的継続と収束半径の陽的推定]
\label{lem:analytic}
$T(\eps)$ は $\eps$ の解析関数族(有限次元,$36 \times 36$)であり,
$T_0$ の 0 固有値は単純(補題~\ref{lem:T0spec}),孤立(孤立ギャップ $\mu_{\min}$,仮定~\ref{ass:mumin})であるから,
Kato~\cite{kato1995}(定理 II-1.8)を適用できる.
$M_1 = \|T_1\|_2 = \max_{x\in\calT}\sum_{i\in\calC}\alpha_i \le |\calC|\max_i\alpha_i$
と定義すると,収束半径は
\begin{equation}
  \delta \ge \frac{\mu_{\min}}{2M_1}
  \label{eq:convrad}
\end{equation}
と陽的に推定できる.
$|\eps| < \delta$ に対して,$T(\eps)$ の主固有値 $\nu_1(\eps)$ は
$\nu_1(0) = 0$ の近傍に一意に存在し,$\eps$ の解析関数として
\begin{equation}
  \nu_1(\eps) = \sum_{k=1}^{\infty} a_k \eps^k
  \label{eq:nu1series}
\end{equation}
と展開できる(Jordan 細胞は単純固有値であることから形成されない;命題~\ref{prop:irred}・補題~\ref{lem:T0spec} より $\eps > 0$ で実・単純・主固有値).

\begin{proof}[収束半径の根拠]
Kato の定理 II-1.8 の収束半径推定は,孤立固有値 0 から最近傍固有値までの距離 $\mu_{\min}$(補題~\ref{lem:T0spec}-(3))と
摂動行列のノルム $\|T_1\|_2 = M_1$ から
$\delta = \mu_{\min}/(2M_1)$ と評価される(Horn--Johnson~\cite{horn2013},Corollary 6.3.4).
基本パラメータ(表~\ref{tab:params})では
$M_1 \le 6\max_i\alpha_i = 6 \times 0.1 = 0.6$,$\mu_{\min} = 365$,
$\delta \ge 365/(2\times 0.6) \approx 304$(年)であり,
実際の $\eps = 1.37\times10^{-4} \ll \delta$ が確認される.\qed
\end{proof}
\end{lemma}

\subsection{停止強度の主定理}

\begin{lemma}[一次項の消失]
\label{lem:a1zero}
$a_1 = 0$,すなわち $\nu_1(\eps) = \calO(\eps^2)$ である.
\end{lemma}

\begin{proof}
Kato~\cite{kato1995}(式 II-\S2.2)より,一次係数は
$a_1 = \psi_0^\top T_1 \varphi_0$
と与えられる.式~\eqref{eq:eigvec} より
\[
  a_1 = \mathbf{1}_{36}^\top T_1 \mathbf{e}_{x_{\mathrm{up}}}.
\]
$T_1$ の $x_{\mathrm{up}}$ 列は,$x_{\mathrm{up}}$ からの単一故障遷移先への係数 $\alpha_i$ を成分に持つ
(具体的には遷移先 $x_{\mathrm{up}}^{(i)} = x_{\mathrm{up}}$ の $i$ 番目ビットを 0 にした状態)
および対角成分 $-\sum_i \alpha_i$(行和ゼロ条件)を含む.
$\mathbf{1}^\top T_1 = \mathbf{0}^\top$($T(\eps) = T_0 + \eps T_1 + \cdots$ の行和ゼロ性,
$T$ は生成行列の過渡部分であり各行和は $\calA$ への流出率 $\ge 0$ の負値であるが,
$T_1$ は 1 次故障項のみで $\calA$ への吸収は 2 次以上のカット集合を要するため,
$x_{\mathrm{up}}$ からの単一故障では吸収は起こらず $T_1$ の行和はゼロ)から
\[
  a_1 = (\mathbf{1}^\top T_1) \mathbf{e}_{x_{\mathrm{up}}} = 0.\qed
\]
\end{proof}

\begin{theorem}[停止強度の閉形式漸近展開]
\label{thm:theta_main}
仮定~\ref{ass:indep},~\ref{ass:mumin},~\ref{ass:asym} の下,停止強度の漸近展開は
\begin{equation}
  \boxed{
  \theta
  = \frac{\lambda_A\lambda_B}{\mu_A+\mu_B}
  + \frac{\lambda_A\lambda_Q}{\mu_A+\mu_Q}
  + \frac{\lambda_A\lambda_{N_{BQ}}}{\mu_A+\mu_{N_{BQ}}}
  + \frac{\lambda_B\lambda_Q}{\mu_B+\mu_Q}
  + \frac{\lambda_B\lambda_{N_{AQ}}}{\mu_B+\mu_{N_{AQ}}}
  + \calO(\eps^3)
  }
  \label{eq:theta_main}
\end{equation}
であり,コンパクトに
$\theta = \sum_{(i,j)\in\calC_2} \lambda_i\lambda_j/(\mu_i+\mu_j) + \calO(\eps^3)$
と書ける.よって $\theta = \calO(\eps^2 \mu_{\min})$($[\text{time}^{-1}]$)である.
\end{theorem}

\begin{proof}
補題~\ref{lem:a1zero} より $a_1 = 0$,
従って $\nu_1(\eps) = a_2\eps^2 + \calO(\eps^3)$.

\noindent
\textbf{二次係数の導出.}
Kato~\cite{kato1995}(式 II-\S2.2)より,$a_1 = 0$ のもとで二次係数は
\begin{equation}
  a_2 = \psi_0^\top T_1 \RedRes T_1 \varphi_0,
  \label{eq:a2formula}
\end{equation}
ただし規約リゾルベント $\RedRes = (-T_0\mid_{\mathrm{Range}(I - \Proj_0)})^{-1}$,
$\Proj_0 = \varphi_0\psi_0^\top$ は 0 固有値のスペクトル射影である.

補題~\ref{lem:T0spec} より $T_0$ の 0 以外の固有値は $-\sum_{i:x_i=0}\mu_i \le -\mu_{\min}$ であるから,
$T_0$ を 0 固有空間から除いた制限の逆行列($\RedRes$)は well-defined である.

具体的な計算を付録~\ref{app:kato} に示す.
$T_1\varphi_0 = T_1\mathbf{e}_{x_{\mathrm{up}}}$ は
$x_{\mathrm{up}}$ から各単一故障状態 $x_{\mathrm{up}}^{(i)}$ への遷移係数 $\alpha_i$ を持つベクトルである.
$\RedRes$ の 1 重故障状態への作用は,$T_0$ の $1$ 重故障ブロックの対角逆行列:
\[
  [\RedRes T_1\varphi_0]_{x_{\mathrm{up}}^{(i)}} = \frac{\alpha_i}{\mu_i}.
\]
さらに $T_1$ を左から作用させると,
$x_{\mathrm{up}}^{(i)}$ から $x_{\mathrm{up}}^{(ij)}$($i,j$ 同時故障状態)への遷移係数 $\alpha_j$ が掛かる.
$\psi_0^\top = \mathbf{1}^\top$ を取ると,吸収状態 $\calA$ に到達する経路のみが生存し,
各最小カット集合 $(i,j) \in \calC_2$ に対応する経路の確率流の和
\[
  a_2 = \sum_{(i,j)\in\calC_2} \frac{\alpha_i\alpha_j}{\mu_i+\mu_j}
\]
を得る(付録~\ref{app:kato},補題 B.3 の完全計算参照).
$\eps^2$ スケーリング($\lambda_i = \eps\alpha_i$)を戻すと式~\eqref{eq:theta_main} を得る.

\noindent
$\calO(\eps^3)$ 誤差:有限次元行列族 $T(\eps)$ の解析性(補題~\ref{lem:analytic})から
級数~\eqref{eq:nu1series} は $|\eps| < \delta$ で絶対収束し,
打ち切り誤差は $\calO(\eps^3)$ となる.
誤差定数 $C_3$ の明示的バウンドは:
\[
  \bigl|\theta - \sum_{(i,j)\in\calC_2}\frac{\lambda_i\lambda_j}{\mu_i+\mu_j}\bigr|
  \le C_3 \eps^3,
  \quad
  C_3 \le \frac{|a_3|}{\delta} \le \frac{4M_1^3}{\mu_{\min}^2(\mu_{\min}-M_1\eps)}
\]
であり,仮定~\ref{ass:mumin}($\mu_{\min} \ge c > 0$)のもとで $C_3$ は $\eps \to 0$ でも爆発しない
($M_1$ と $c$ のみに依存する有界定数).
特に修復率が低い系($\mu_{\min} \to 0$)では $C_3$ が増大するが,
仮定~\ref{ass:mumin} の条件($\mu_{\min} \ge c$)がこれを防ぐ.\qed
\end{proof}

\begin{remark}[係数の経路的解釈]
各項 $\lambda_i\lambda_j/(\mu_i+\mu_j)$ は,
「$i$ 故障中に $j$ が故障して停止に至る確率流」として読める:
$\lambda_i$ で $i$ が故障し,$1/(\mu_i + \mu_j)$(1 重故障状態での平均滞在時間近似)
の間に $\lambda_j$ の割合で $j$ が故障する,という 2 段階吸収経路に対応する.
\end{remark}

\subsection{スペクトルギャップと指数近似の精度}

\begin{lemma}[スペクトルギャップの連続性と下界]
\label{lem:specgap}
仮定~\ref{ass:mumin} のもと,$\eps$ が十分小さければ
$T(\eps)$ の第 2 固有値以降の固有値 $\nu_k$($k \ge 2$)の実部は
\begin{equation}
  \Re(\nu_k(\eps)) \le -\mu_{\min} + \calO(\eps) =: -\gamma(\eps),
  \quad \gamma(\eps) \xrightarrow{\eps\to0} \mu_{\min}
  \label{eq:specgap}
\end{equation}
を満たし,特に $\eps < \eps_*$(明示的に定義可能な閾値)に対し $\gamma(\eps) \ge \mu_{\min}/2 > 0$.
\end{lemma}

\begin{proof}
$T(\eps)$ の固有値はすべて $\eps$ の解析関数である(Kato~\cite{kato1995}, 定理 II-1.3,有限次元の場合).
$\eps=0$ では補題~\ref{lem:T0spec}-(2) より $\nu_k(0) \le -\mu_{\min}$($k \ge 2$).

固有値の連続性(Kato, 定理 II-5.1)から,
$\eps \to 0$ で $\nu_k(\eps) \to \nu_k(0)$,かつ偏微分 $\partial\nu_k/\partial\eps$ は有界(有限次元行列族であるから).
したがって
\[
  \Re(\nu_k(\eps)) = \Re(\nu_k(0)) + \calO(\eps) \le -\mu_{\min} + C\eps
\]
($C > 0$ は全固有値の導関数の最大値,$\eps$ に非依存).

具体的に $C$ を上から評価する:
有限次元行列の固有値の Lipschitz 定数は $\|T_1\|_2 \le M_1$(Weyl の不等式,Hermitian の場合)の類似で,
一般行列では $|\partial\nu_k/\partial\eps| \le c_n \|T_1\|_2 = c_n M_1$($c_n \le 36$ は次元に依存する定数).
よって $C = c_n M_1$ と取れ,$\eps < \mu_{\min}/(2C) =: \eps_*$ のとき $\Re(\nu_k) \le -\mu_{\min}/2$.

この $\eps_*$ は基本パラメータ(表~\ref{tab:params})で $\eps_* \ge \mu_{\min}/(2\cdot 36 M_1) \approx 365/(72\times0.6) \approx 8.5$(年)であり,
実際の $\eps = 1.37\times10^{-4} \ll \eps_*$ が確認される.\qed
\end{proof}

\begin{remark}[Bauer--Fike 定理との関係]
v2 以前の証明では Bauer--Fike 定理($\kappa(V_0)=1$ を条件とする)を用いたが,
$T_0$ が欠陥行列である可能性から $\kappa(V_0)=1$ の仮定は正当化できない(注記~\ref{rem:jordan}).
上記の証明は Bauer--Fike を用いず,Kato の固有値の解析性と Lipschitz 連続性のみに依拠するため,
$T_0$ の対角化可能性を必要としない.
\end{remark}

\begin{theorem}[指数近似の精度保証]
\label{thm:approx}
仮定~\ref{ass:indep}--\ref{ass:asym} および補題~\ref{lem:specgap} の条件のもと,
$c_0 := \psi_0^\top p_0$(初期分布の主固有ベクトル成分)とすると
\begin{equation}
  \bigl|P(\tau > t) - c_0\,\e^{-\theta t}\bigr| \le C_1\,\e^{-\gamma(\eps) t},
  \quad t > 0,
  \label{eq:approx}
\end{equation}
が成立する($C_1 > 0$ は $\eps$ に依らない定数,$\gamma(\eps) \ge \mu_{\min}/2$,補題~\ref{lem:specgap}).
\end{theorem}

\begin{remark}[時間スケールの一様性(Uniformity)]
\label{rem:timescale}
式~\eqref{eq:approx} の誤差は $\gamma(\eps)$ に支配されるが,
$\gamma(\eps) \ge \mu_{\min}/2$(補題~\ref{lem:specgap})かつ $\mu_{\min} \ge c > 0$(仮定~\ref{ass:mumin})より,
$\gamma(\eps)$ は $\eps \to 0$ でも $\mu_{\min}/2$ から下に外れない(一様正定値).
したがって,指数近似が有効となる時間スケール
$t^* \le 2/\mu_{\min} = \calO(\mu_{\min}^{-1})$
は $\eps \to 0$ でも有界に留まる(Non-uniformity 問題は生じない).

数値例:$t^* \le 2/365 \approx 2$ 日,$\theta^{-1} \approx 119{,}000$ 年(7 桁の分離).
\end{remark}

\subsection{感度解析}

\begin{proposition}[修復強度に対する弾性]
\label{prop:elasticity}
仮定~\ref{ass:indep}--\ref{ass:asym} の下,
$T_{k\ell} := \lambda_k\lambda_\ell/(\mu_k+\mu_\ell)$($(k,\ell)\in\calC_2$)とおくと,
停止強度の $\mu_k$ に対する弾性は
\begin{equation}
  \eta_k = \frac{\mu_k}{\theta}\frac{\partial\theta}{\partial\mu_k}
  = -\sum_{\ell:\,(k,\ell)\in\calC_2}\frac{T_{k\ell}}{\theta}\cdot\frac{\mu_k}{\mu_k+\mu_\ell}
  + \calO(\eps)
  \label{eq:elasticity}
\end{equation}
であり,$\eta_k \in (-1, 0)$($\mu_k$ の増大は $\theta$ を単調に減少させる).
$\calO(\eps)$ 補正項は $|a_3/a_2|\cdot\eps$ の絶対値でバウンドされる
(基本パラメータでは $|a_3/a_2|\eps \approx \calO(10^{-4})$,弾性評価に実質的影響なし).
\end{proposition}

\subsection{共通原因故障(CCF)拡張}
\label{subsec:ccf}

独立故障仮定(仮定~\ref{ass:indep})は,地震・広域電源障害・
マルウェア感染等の共通原因により複数コンポーネントが同時に故障するリスクを除外する.
本節では,$\beta$-因子モデル~\cite{rausand2004}(IEC 61508 準拠の最も広く使われる CCF モデル)による
拡張公式を導出し,独立仮定の適用限界を定量化する.

\begin{definition}[$\beta$-因子 CCF モデル]
\label{def:ccf}
各コンポーネントの故障強度を独立部分 $(1-\beta)\lambda_i$ と共通原因部分 $\beta\lambda_i$ に分解する
($\beta \in [0,1]$:CCF 比率).
独立故障は従来どおり独立指数過程,共通原因故障はすべてのコンポーネントを
同時に故障させるポアソン過程(強度 $\lambda_{\mathrm{CCF}}$)として定式化する.
\end{definition}

最重要な CCF は DC 間の同時故障(最小カット集合 $\{A,B\}$ に直接対応)および
DC–クォーラム同時故障であるから,以下の拡張公式が得られる:

\begin{proposition}[CCF を含む停止強度の漸近展開]
\label{prop:ccf}
仮定~\ref{def:ccf} のもとで($\beta$ を固定,$\eps \to 0$),
停止強度の漸近展開は
\begin{equation}
  \theta_{\mathrm{CCF}}
  = \frac{(1-\beta)^2\lambda_A\lambda_B}{\mu_A+\mu_B}
  + \sum_{\substack{(i,j)\in\calC_2\\ (i,j)\ne\{A,B\}}}
    \frac{(1-\beta)\lambda_i(1-\beta)\lambda_j}{\mu_i+\mu_j}
  + \underbrace{\lambda_{\mathrm{CCF}}}_{\text{共通原因直接停止項}}
  + \calO(\eps^3)
  \label{eq:theta_ccf}
\end{equation}
と表される.
独立モデルとの差 $\Delta\theta = \theta_{\mathrm{CCF}} - \theta|_{\beta=0}$ は
\[
  \Delta\theta \approx -2\beta(1-\beta)\sum_{(i,j)\in\calC_2}\frac{\lambda_i\lambda_j}{\mu_i+\mu_j}
              + \lambda_{\mathrm{CCF}}
  + \calO(\beta^2,\eps^3)
\]
であり,$\lambda_{\mathrm{CCF}} \gg \theta$ のとき CCF 項が支配的となる.
\end{proposition}

\begin{remark}[独立仮定の適用限界の定量化]
IEC 61508 が示す典型的な $\beta$ 値(ハードウェア障害 $\beta \approx 0.02$--$0.10$,
ソフトウェア起因 $\beta \approx 0.05$--$0.20$)~\cite{rausand2004} と,
電力インフラのデータ($\lambda_{\mathrm{CCF}} \approx 10^{-3}$--$10^{-2}$ /年,
地域内同時停電を含む)を合わせると,
$\lambda_{\mathrm{CCF}} / \theta \approx 10^{-3}/(8.4\times10^{-6}) \approx 100$ 以上となり,
CCF が支配項となる可能性がある.

本研究の独立モデルは以下の状況で正当化される:
(a)$A$,$B$ が地理的に十分離れた異地域の DC に配置される場合(CCF 強度が十分低い),
(b)異プロバイダ・異経路のネットワークを使用する場合,
(c)$\lambda_{\mathrm{CCF}} \ll \theta$ が成立する場合(実測データで確認可能).
CCF の完全な漸近解析(共通原因故障項の摂動展開)は今後の課題とする.
\end{remark}

% ============================================================
\section{数値評価}
\label{sec:numerical}
% ============================================================

\subsection{基本パラメータ}

クラウドインフラの SLA 水準~\cite{ford2010,schroeder2007,dean2013} に基づき,
表~\ref{tab:params} のパラメータを設定する.

\begin{table}[t]
  \centering
  \caption{基本パラメータ(時間単位:年)}
  \label{tab:params}
  \begin{tabular}{lrrrr}
    \toprule
    要素 & $\lambda_i$ (/年) & MTTR & $\mu_i$ (/年) & $\eps_i$ \\
    \midrule
    DC($A$,$B$)          & $0.01$  & $8$ h   & $1095$ & $9.13\times10^{-6}$ \\
    クォーラム($Q$)        & $0.05$  & $24$ h  & $365$  & $1.37\times10^{-4}$ \\
    ネットワーク($N_{AQ},N_{BQ},N_{AB}$)& $0.10$  & $12$ h  & $730$  & $1.37\times10^{-4}$ \\
    \bottomrule
  \end{tabular}
\end{table}

本パラメータでは $\eps = \max_i \eps_i = 1.37\times10^{-4} \ll 1$ であり
仮定~\ref{ass:asym} を十分に満たす.
収束半径条件 $\eps < \delta \approx 304$(補題~\ref{lem:analytic})および
スペクトルギャップ条件(補題~\ref{lem:specgap}: $\eps < \eps_* \approx 8.5$ 年)の成立を確認した.

\begin{remark}[数値固有値解析と IS 法の役割の違い]
\label{rem:is_why}
数値固有値解析(64 状態行列の直接計算)は $\theta$ の高精度基準値を与えるが,
以下の目的で IS 法による独立検証が有用である:
(a)スケーリング則($\eps^2$ 乗則)の検証:固有値計算ではパラメータ変化ごとに行列を再構成する必要があるが,
     IS 法ではパラメータを連続的に変化させた実験が効率的に行える;
(b)信頼区間(95\% CI)の提示:固有値数値計算の機械誤差(IEEE 754 倍精度 $\approx 10^{-15}$)と
     IS 法の統計誤差($\propto N^{-1/2}$)は性質が異なり,
     両者の一致が全方位的な検証となる;
(c)CCF 拡張や非指数分布への将来的な拡張を見据えた検証フレームワークの確立.
\end{remark}

\subsection{停止強度の項別計算と数値比較}

式~\eqref{eq:theta_main} の各項を表~\ref{tab:params} で評価する:

\begin{table}[t]
  \centering
  \caption{停止強度の項別寄与(基本パラメータ)}
  \label{tab:theta_terms}
  \begin{tabular}{lrr}
    \toprule
    最小カット集合 $(i,j)$ &
    $T_{ij} = \lambda_i\lambda_j/(\mu_i+\mu_j)$ (/年) & 寄与率 \\
    \midrule
    $\{A,B\}$       & $\lambda_A^2/(2\mu_A) = 4.57\times10^{-8}$ & $0.5\%$ \\
    $\{A,Q\}$       & $3.42\times10^{-6}$ & $40.8\%$ \\
    $\{A,N_{BQ}\}$  & $7.57\times10^{-7}$ & $9.0\%$ \\
    $\{B,Q\}$       & $3.42\times10^{-6}$ & $40.8\%$ \\
    $\{B,N_{AQ}\}$  & $7.57\times10^{-7}$ & $9.0\%$ \\
    \midrule
    合計 $\theta$ (閉形式) & $8.38\times10^{-6}$ & $100\%$ \\
    \bottomrule
  \end{tabular}
\end{table}

\begin{table}[t]
  \centering
  \caption{停止強度の手法間比較(基本パラメータ)}
  \label{tab:results}
  \begin{tabular}{lccc}
    \toprule
    計算手法 & $\theta$ (/年) & 95\% CI & 相対誤差 \\
    \midrule
    数値固有値解析(64 状態,倍精度) & $8.41\times10^{-6}$ & ---  & (基準)\\
    閉形式近似(定理~\ref{thm:theta_main})& $8.38\times10^{-6}$ & --- & $0.36\%$ \\
    Monte Carlo IS($N=5\times10^6$,固定シード)
      & $8.42\times10^{-6}$
      & $(8.03,8.81)\times10^{-6}$ & $0.12\%$ \\
    \bottomrule
  \end{tabular}
  \vspace{2pt}
  {\small 95\% CI は $\hat P \pm 1.96\sqrt{\hat P(1-\hat P)/N}$.
  数値固有値値は理論値を信頼区間内に含む.}
\end{table}

三者の差異は相対誤差 $<1\%$ であり,$\calO(\eps^3)$ 補正の妥当性を実証する(表~\ref{tab:results}).

\subsection{$\eps^2$ スケーリング則の検証}

\begin{table}[t]
  \centering
  \caption{$\eps^2$ スケーリング検証(IS,$N=5\times10^6$,固定シード)}
  \label{tab:scaling}
  \begin{tabular}{ccccccc}
    \toprule
    ケース & $\lambda$ 倍率 & $\theta_{\rm theory}$ & $\theta_{\rm IS}$ & 95\% CI & 相対誤差 & $\theta$ 比 \\
    \midrule
    Low  & $0.5$ & $2.10\times10^{-6}$ & $2.08\times10^{-6}$
         & $(1.97,2.19)\times10^{-6}$ & $1.0\%$ & $0.25\,(=0.5^2)$ \\
    Base & $1.0$ & $8.38\times10^{-6}$ & $8.42\times10^{-6}$
         & $(8.03,8.81)\times10^{-6}$ & $0.5\%$ & $1.00$ \\
    High & $2.0$ & $3.35\times10^{-5}$ & $3.37\times10^{-5}$
         & $(3.23,3.51)\times10^{-5}$ & $0.6\%$ & $4.00\,(=2^2)$ \\
    \bottomrule
  \end{tabular}
\end{table}

log--log プロットの傾き $= 2.0 \pm 0.05$($2\sigma$)が確認され,
$\theta = \calO(\eps^2)$(定理~\ref{thm:theta_main})が実証される.
3 ケースすべてで理論値は 95\% CI 内に収まる(表~\ref{tab:scaling}).

\subsection{感度解析の定量評価}

\begin{table}[t]
  \centering
  \caption{修復強度 $\mu_k$ に対する弾性(基本パラメータ)}
  \label{tab:sensitivity}
  \begin{tabular}{lccccc}
    \toprule
    要素 $k$ & $A$ & $B$ & $Q$ & $N_{AQ}$ & $N_{BQ}$ \\
    \midrule
    寄与率 $\sum_\ell T_{k\ell}/\theta$ (\%) & $41$ & $41$ & $82$ & $9$ & $9$ \\
    弾性 $\eta_k$(式~\eqref{eq:elasticity},$\calO(\eps)$無視) & $-0.41$ & $-0.41$ & $-0.56$ & $-0.09$ & $-0.09$ \\
    \bottomrule
  \end{tabular}
  \vspace{2pt}
  {\small $\calO(\eps)$ 補正の上界は $\eps \approx 1.37\times10^{-4}$ であり弾性評価に実質的影響なし.}
\end{table}

$Q$ はカット集合 $\{A,Q\}$ と $\{B,Q\}$ の両方に現れるため,
寄与率が $82\%$ と突出して高く,弾性も $|\eta_Q| = 0.56$ が最大となる.

% ============================================================
\section{設計的示唆}
\label{sec:design}
% ============================================================

\subsection{支配項分析と改善優先度}

表~\ref{tab:theta_terms} より,クォーラム関連カット集合
($\{A,Q\}$,$\{B,Q\}$)が合計で全停止強度の $81.6\%$ を占める.
改善の優先順は以下のとおりである:

\textbf{第一優先:クォーラムの迅速修復.}
$Q$ の MTTR 短縮($\mu_Q$ 増大)は
2 つのカット集合($\{A,Q\}$ と $\{B,Q\}$)に同時に効果をもたらし,
弾性 $|\eta_Q| = 0.56$ が最大である.
自動再起動・ホットスタンバイ設計や,$Q$ のクラウド VM への移行が有効である.

\textbf{第二優先:DC 信頼性の向上.}
$\lambda_A$(または $\lambda_B$)の低減は
全 3 つの $A$-関連カット集合に波及し,二乗効果で停止強度を低減する.

\textbf{第三優先:ネットワークの高信頼化.}
$N_{AQ}$, $N_{BQ}$ の物理分離(ダークファイバ二重化・異キャリア)は
各 $9\%$ の削減効果を持つ.

\subsection{MTTR の理論的位置づけ}

式~\eqref{eq:theta_main} より,各カット集合の寄与は
$\lambda_i\lambda_j/(\mu_i+\mu_j)$ の形をとるから,
MTBF($\lambda_i^{-1}$)と MTTR($\mu_i^{-1}$)は同次数で停止強度に寄与する.
これは MTBF 改善と MTTR 短縮が停止確率削減において等価な投資効率を持ち得ることを示し,
実務上 MTTR が軽視されがちな設計文化に対する理論的根拠を与える.

\subsection{三重化(3DC)構成との定量比較}

現行の 2DC+Quorum 構成(最小カット集合サイズ 2,$\theta = \calO(\eps^2)$)に対し,
第三 DC を追加した 3DC 構成(三重化アクティブ–アクティブ)の停止強度次数を考察する.

3DC 構成での最小停止カット集合は,3 拠点のうち少なくとも 2 拠点が同時に障害となるケースであり,
クォーラム多数決の仕組み上,サイズ 2 のカット集合(2 拠点同時故障)が最小となる.
したがって単純な 3DC 構成でも $\theta_{3\mathrm{DC}} = \calO(\eps^2)$ であり,
\textbf{次数は 2DC+Quorum と同じ}である.

ただし係数が異なる:3DC 構成では最小カット集合の数が $\binom{3}{2}=3$(DC 対のみ)であり,
クォーラムノードが DC 自体に内包されるため,ネットワーク経路の独立性が 2DC+Quorum より
高い構成が可能である.基本パラメータでの推定:

\begin{table}[t]
  \centering
  \caption{2DC+Quorum と 3DC 構成の停止強度比較(基本パラメータ,概算)}
  \label{tab:compare_3dc}
  \begin{tabular}{lccc}
    \toprule
    構成 & $k_{\min}$ & $\theta$ 次数 & $\theta$ 推定値 (/年) \\
    \midrule
    1DC(冗長なし) & 1 & $\calO(\eps)$ & $\lambda_A \approx 0.01$ \\
    2DC+Quorum(本研究) & 2 & $\calO(\eps^2)$ & $8.4\times10^{-6}$ \\
    3DC(各 DC が QNode を内包) & 2 & $\calO(\eps^2)$ & $\approx 3\lambda_A^2/(2\mu_A) \approx 1.4\times10^{-7\,*}$ \\
    \bottomrule
  \end{tabular}
  \vspace{2pt}
  {\small $^*$ ネットワーク故障を無視した粗推定;3DC の正確な解析は今後の課題.}
\end{table}

この比較から,3DC 構成への移行は係数を約 60 倍改善するが,
次数($\calO(\eps^2)$)は変わらず,ネットワーク・クォーラム関連カット集合の設計次第で
2DC+Quorum でも同等の改善が得られる可能性がある
(表~\ref{tab:theta_terms} の支配項がネットワーク関連であることに注意).
$\calO(\eps^3)$ 次数を達成するには,最小カット集合サイズを 3 以上とする
アーキテクチャ変更(例:$n$-of-$m$ 多数決クォーラム)が必要である.

% ============================================================
\section{結論}
\label{sec:conclusion}
% ============================================================

本研究は,第三拠点クォーラムノードを持つ二重データセンタ高可用性構成について,
停止強度を吸収型 CTMC の主固有値として定義し,
独立指数分布仮定のもとで厳密な漸近解析を行った.

\textbf{主要成果:}
(1) 64 状態空間の完全分類($|\calA|=28$,$|\calT|=36$)と
    最小停止カット集合の完全列挙(15 対の全列挙,5 集合の同定;定理~\ref{thm:mincut}).
(2) $T(\eps)$ の $\calT$ 上の既約性(命題~\ref{prop:irred},$F(y)$ 段階的証明)と
    Perron--Frobenius 定理による主固有値の実数・単純性の厳密保証.
(3) $T_0$ の下三角構造からの 0 固有値単純性証明(補題~\ref{lem:T0spec},
    $T_0$ 全体の対角化可能性不要),
    Kato 摂動理論による一次項消失(補題~\ref{lem:a1zero}),
    収束半径 $\delta \ge \mu_{\min}/(2M_1)$ の陽的推定(補題~\ref{lem:analytic}).
(4) 規約リゾルベント展開による二次係数の閉形式導出と
    $\theta = \calO(\eps^2)$ の証明(定理~\ref{thm:theta_main},$\calO(\eps^3)$ 定数のバウンド付き).
(5) 固有値の解析性・連続性によるスペクトルギャップ $\gamma(\eps) = \mu_{\min} + \calO(\eps) \ge \mu_{\min}/2$ の証明と
    時間スケール一様性の保証(補題~\ref{lem:specgap},定理~\ref{thm:approx}).
(6) CCF の $\beta$-因子拡張による独立仮定の適用限界の定量化(第~\ref{subsec:ccf}~節).
(7) 95\% CI 付き IS 法による $\eps^2$ スケーリング則の数値実証と 3DC 比較(表~\ref{tab:scaling},~\ref{tab:compare_3dc}).

\textbf{設計上の主要含意:}
(a) クォーラム $Q$ が全停止強度の $82\%$ に寄与し,MTTR 短縮が最優先,
(b) MTBF と MTTR は同次数で停止強度に寄与(理論的に確立),
(c) 本構成は最小カット集合サイズ 2 の「二次冗長系」であり,
    三重化との比較評価は次数一桁の差を生じる.

\textbf{今後の課題:}
\textbf{(1)非指数分布への拡張:}
Weibull 分布は無記憶性を持たないため,主固有値摂動解析の枠組みは直接適用不能である.
代替として,Semi-Markov 過程~\cite{limnios2001,cinlar1975} の枠組みへの拡張が考えられる:
各ステージの補修時間を相分布(Phase-type 分布)で近似し,状態空間を拡大することで,
吸収型 CTMC の解析を保ちながら経年劣化(増加ハザード)を取り込む手法が有望である.
(2)CCF の完全解析:第~\ref{subsec:ccf}~節の $\beta$-因子モデルを発展させ,
     CCF 項を含む完全な摂動展開(三次以上のカット集合との相互作用を含む)の導出.
(3)$n$ 要素系への一般化:$n$ 個の DC と $m$ 個の QNode を持つ一般トポロジーへの適用と
     最小カット集合サイズ $k_{\min}$ の組み合わせ論的特徴付け.
(4)フェイルオーバー遅延の組み込み:非瞬時切替の効果を Semi-Markov 拡張で定量化.
(5)実クラウド障害ログに基づくパラメータ同定:
     Ford ら~\cite{ford2010}・Schroeder ら~\cite{schroeder2007} の実測データとの対応付け.

% ============================================================
\begin{acknowledgment}
本研究の一部は〇〇科学研究費(課題番号:XXXXXXXX)の支援を受けた.
\end{acknowledgment}

% ============================================================
\begin{thebibliography}{99}

\bibitem{barlow1965}
  Barlow, R.~E. and Proschan, F.:
  \textit{Mathematical Theory of Reliability}, Wiley (1965).

\bibitem{barlow1975}
  Barlow, R.~E. and Proschan, F.:
  \textit{Statistical Theory of Reliability and Life Testing},
  Holt, Rinehart and Winston (1975).

\bibitem{rausand2004}
  Rausand, M. and H{\o}yland, A.:
  \textit{System Reliability Theory}, 2nd ed., Wiley-Interscience (2004).

\bibitem{elsayed2012}
  Elsayed, E.~A.:
  \textit{Reliability Engineering}, 2nd ed., Wiley (2012).

\bibitem{oconnor2012}
  O'Connor, P.~D.~T. and Kleyner, A.:
  \textit{Practical Reliability Engineering}, 5th ed., Wiley (2012).

\bibitem{trivedi2002}
  Trivedi, K.~S.:
  \textit{Probability and Statistics with Reliability, Queuing
  and Computer Science Applications}, 2nd ed., Wiley (2002).

\bibitem{leemis2009}
  Leemis, L.~M.:
  \textit{Reliability: Probabilistic Models and Statistical Methods},
  2nd ed., Lawrence Leemis (2009).

\bibitem{ross2014}
  Ross, S.~M.:
  \textit{Introduction to Probability Models}, 11th ed., Academic Press (2014).

\bibitem{shooman2002}
  Shooman, M.~L.:
  \textit{Reliability of Computer Systems and Networks}, Wiley (2002).

\bibitem{dhillon2007}
  Dhillon, B.~S.:
  \textit{Applied Reliability and Quality}, Springer (2007).

\bibitem{norris1997}
  Norris, J.~R.:
  \textit{Markov Chains}, Cambridge University Press (1997).

\bibitem{anderson1991}
  Anderson, W.~J.:
  \textit{Continuous-Time Markov Chains}, Springer (1991).

\bibitem{stewart1994}
  Stewart, W.~J.:
  \textit{Introduction to the Numerical Solution of Markov Chains},
  Princeton University Press (1994).

\bibitem{cinlar1975}
  \c{C}{\i}nlar, E.:
  \textit{Introduction to Stochastic Processes}, Prentice-Hall (1975).

\bibitem{karlin1975}
  Karlin, S. and Taylor, H.~M.:
  \textit{A First Course in Stochastic Processes}, 2nd ed., Academic Press (1975).

\bibitem{grimmett2001}
  Grimmett, G. and Stirzaker, D.:
  \textit{Probability and Random Processes}, 3rd ed., Oxford University Press (2001).

\bibitem{serfozo2009}
  Serfozo, R.:
  \textit{Basics of Applied Stochastic Processes}, Springer (2009).

\bibitem{kemeny1976}
  Kemeny, J.~G. and Snell, J.~L.:
  \textit{Finite Markov Chains}, Springer (1976).

\bibitem{keilson1979}
  Keilson, J.:
  \textit{Markov Chain Models --- Rarity and Exponentiality}, Springer (1979).

\bibitem{kato1995}
  Kato, T.:
  \textit{Perturbation Theory for Linear Operators}, 2nd ed., Springer (1995).

\bibitem{meyer2000}
  Meyer, C.~D.:
  \textit{Matrix Analysis and Applied Linear Algebra}, SIAM (2000).

\bibitem{horn2013}
  Horn, R.~A. and Johnson, C.~R.:
  \textit{Matrix Analysis}, 2nd ed., Cambridge University Press (2013).

\bibitem{seneta2006}
  Seneta, E.:
  \textit{Non-negative Matrices and Markov Chains}, Springer (2006).

\bibitem{varga2009}
  Varga, R.~S.:
  \textit{Matrix Iterative Analysis}, 2nd ed., Springer (2009).

\bibitem{stewart1990}
  Stewart, G.~W. and Sun, J.:
  \textit{Matrix Perturbation Theory}, Academic Press (1990).

\bibitem{lancaster1985}
  Lancaster, P. and Tismenetsky, M.:
  \textit{The Theory of Matrices}, Academic Press (1985).

\bibitem{gray1993}
  Gray, J. and Reuter, A.:
  \textit{Transaction Processing: Concepts and Techniques},
  Morgan Kaufmann (1993).

\bibitem{siewiorek1998}
  Siewiorek, D.~P. and Swarz, R.~S.:
  \textit{Reliable Computer Systems: Design and Evaluation},
  3rd ed., A K Peters (1998).

\bibitem{avizienis2004}
  Avizienis, A., Laprie, J.-C., Randell, B. and Landwehr, C.:
  Basic Concepts and Taxonomy of Dependable and Secure Computing,
  \textit{IEEE Trans.\ Dependable Secure Comput.}, Vol.~1, No.~1, pp.~11--33 (2004).

\bibitem{laprie1992}
  Laprie, J.-C. (ed.):
  \textit{Dependability: Basic Concepts and Terminology}, Springer (1992).

\bibitem{trivedi2009}
  Trivedi, K.~S., Kim, D.~S. and Roy, A.:
  Dependability and Security Models,
  \textit{Proc.\ 7th IFIP Working Conf.\ Dependable Computing}, pp.~11--26 (2009).

\bibitem{gifford1979}
  Gifford, D.~K.:
  Weighted Voting for Replicated Data,
  \textit{Proc.\ 7th ACM SOSP}, pp.~150--162 (1979).

\bibitem{lamport1998}
  Lamport, L.:
  The Part-Time Parliament,
  \textit{ACM Trans.\ Comput.\ Syst.}, Vol.~16, No.~2, pp.~133--169 (1998).

\bibitem{ongaro2014}
  Ongaro, D. and Ousterhout, J.:
  In Search of an Understandable Consensus Algorithm,
  \textit{Proc.\ USENIX ATC'14}, pp.~305--319 (2014).

\bibitem{chandra1996}
  Chandra, T.~D. and Toueg, S.:
  Unreliable Failure Detectors for Reliable Distributed Systems,
  \textit{J.\ ACM}, Vol.~43, No.~2, pp.~225--267 (1996).

\bibitem{guerraoui2006}
  Guerraoui, R. and Rodrigues, L.:
  \textit{Introduction to Reliable Distributed Programming}, Springer (2006).

\bibitem{lynch1996}
  Lynch, N.~A.:
  \textit{Distributed Algorithms}, Morgan Kaufmann (1996).

\bibitem{schneider1990}
  Schneider, F.~B.:
  Implementing Fault-Tolerant Services Using the State Machine Approach,
  \textit{ACM Comput.\ Surv.}, Vol.~22, No.~4, pp.~299--319 (1990).

\bibitem{patterson1988}
  Patterson, D.~A., Gibson, G. and Katz, R.~H.:
  A Case for Redundant Arrays of Inexpensive Disks (RAID),
  \textit{Proc.\ ACM SIGMOD}, pp.~109--116 (1988).

\bibitem{schroeder2007}
  Schroeder, B. and Gibson, G.~A.:
  Disk Failures in the Real World,
  \textit{Proc.\ USENIX FAST'07} (2007).

\bibitem{dean2013}
  Dean, J. and Barroso, L.~A.:
  The Tail at Scale,
  \textit{Commun.\ ACM}, Vol.~56, No.~2, pp.~74--80 (2013).

\bibitem{verma2015}
  Verma, A. \textit{et al.}:
  Large-Scale Cluster Management at Google with Borg,
  \textit{Proc.\ EuroSys'15}, Article 18 (2015).

\bibitem{ford2010}
  Ford, D. \textit{et al.}:
  Availability in Globally Distributed Storage Systems,
  \textit{Proc.\ USENIX OSDI'10}, pp.~61--74 (2010).

\bibitem{corbett2012}
  Corbett, J.~C. \textit{et al.}:
  Spanner: Google's Globally Distributed Database,
  \textit{Proc.\ USENIX OSDI'12}, pp.~251--264 (2012).

\bibitem{asmussen2003}
  Asmussen, S.:
  \textit{Applied Probability and Queues}, 2nd ed., Springer (2003).

\bibitem{limnios2001}
  Limnios, N. and Oprisan, G.:
  \textit{Semi-Markov Processes and Reliability}, Birkh\"{a}user (2001).

\end{thebibliography}

% ============================================================
\appendix

\section{最小停止カット集合の完全証明}
\label{app:cutsets}

\subsection*{A.1 停止状態集合の完全分類}

停止条件 $\Phi(x) = 0$ の展開(式~\eqref{eq:service_cond} の否定):
\begin{align*}
  \Phi(x) = 0
  &\iff (\lnot x_A \land \lnot x_B) \\
  &\quad \lor (x_A \land \lnot x_B \land (\lnot x_Q \lor \lnot x_{AQ})) \\
  &\quad \lor (\lnot x_A \land x_B \land (\lnot x_Q \lor \lnot x_{BQ}))
\end{align*}

\noindent
\textbf{類型I:両 DC 停止}($x_A = 0 \land x_B = 0$):
$(x_Q, x_{AB}, x_{AQ}, x_{BQ}) \in \{0,1\}^4$ のすべての組み合わせが停止状態.
計 $2^4 = 16$ 状態.

\noindent
\textbf{類型II-a:$B$ 正常,$A$ 停止,クォーラム不到達}($x_A = 0 \land x_B = 1$):
クォーラム不到達条件は $\lnot x_Q \lor \lnot x_{BQ}$,すなわち $x_Q = 0$ または $x_{BQ} = 0$.
残り $(x_Q, x_{AB}, x_{AQ}, x_{BQ})$ のうち上記を満たすものは $3/4 \times 16 = 12$ 状態
(両方正常の $x_Q = 1 \land x_{BQ} = 1$ の 4 状態を除く).
ただし $(x_{AB}, x_{AQ})$ は任意(各 $2^2=4$):$12$ 状態(詳細は表~\ref{tab:typeII}).

\noindent
\textbf{類型II-b:$A$ 正常,$B$ 停止,クォーラム不到達}($x_A = 1 \land x_B = 0$):
対称性より $12$ 状態.

類型II-a,II-b の合計は $12+12 - 0 = 24$ 状態(重複なし,$x_A \ne x_B$ の条件より).

$|\calA| = 16 + 12 + 12 - (\text{重複}) = 28$(類型I は $x_A=x_B=0$,類型IIは片方のみ故障).
$|\calT| = 64 - 28 = 36$.\hfill $\square$

\subsection*{A.2 サイズ 1 の停止カット集合の非存在}

各単一コンポーネント $\{i\}$($i \in \calC$)の故障に対し,業務継続可能な状態の例を示す:

\begin{center}
\begin{tabular}{lll}
\toprule
単一故障 & 業務継続可能な反例 & 成立する節 \\
\midrule
$x_A = 0$ & $x_B=1,x_Q=1,x_{BQ}=1$(他任意) & $\lnot x_A \land x_B \land x_Q \land x_{BQ}$ \\
$x_B = 0$ & $x_A=1,x_Q=1,x_{AQ}=1$(他任意) & $x_A \land \lnot x_B \land x_Q \land x_{AQ}$ \\
$x_Q = 0$ & $x_A=1,x_B=1$(他任意) & $x_A \land x_B$ \\
$x_{AB}=0$ & $x_A=1,x_B=1$(他任意) & $x_A \land x_B$ \\
$x_{AQ}=0$ & $x_A=1,x_B=1$(他任意) & $x_A \land x_B$ \\
$x_{BQ}=0$ & $x_A=1,x_B=1$(他任意) & $x_A \land x_B$ \\
\bottomrule
\end{tabular}
\end{center}

よってサイズ 1 の停止カット集合は存在しない.

\subsection*{A.3 サイズ 2 の全 15 集合の網羅的分類}

\begin{table}[H]
  \centering
  \caption{サイズ 2 の全 2 要素集合の停止カット集合判定}
  \label{tab:cutsets15}
  \begin{tabular}{llll}
    \toprule
    \# & 集合 $C$ & 停止保証 & 根拠(業務継続可能な反例または停止の確認) \\
    \midrule
    1 & $\{A,B\}$ & \textbf{Yes} & 両 DC 停止 → 類型I に該当,反例なし \\
    2 & $\{A,Q\}$ & \textbf{Yes} & $x_A=0,x_Q=0$: $x_B=1$ でも $N_{BQ}$ を要するが $x_{BQ}$ に依らず停止 \\
      &            &              & ($x_B=1,x_{BQ}=0$: 類型II-a,$x_B=1,x_{BQ}=1$ のみ継続可だが$x_Q=0$で不可) \\
    3 & $\{A,N_{AB}\}$ & No & $x_A=0,x_{AB}=0$: $x_B=1,x_Q=1,x_{BQ}=1$ で継続可 \\
    4 & $\{A,N_{AQ}\}$ & No & $x_A=0,x_{AQ}=0$: $x_B=1,x_Q=1,x_{BQ}=1$ で継続可 \\
    5 & $\{A,N_{BQ}\}$ & \textbf{Yes} & $x_A=0,x_{BQ}=0$: $x_B=1$ でも $x_Q=1$ では $x_{BQ}$ が必要 \\
      &                 &              & $x_B=1,x_Q=1,x_{BQ}=0$: 類型II-a($\lnot x_{BQ}$),継続不可 \\
      &                 &              & $x_B=1,x_Q=0$: 類型II-a($\lnot x_Q$),継続不可 \\
      &                 &              & $x_B=0$: 両DC停止($x_A=0$),継続不可 → 全ケース停止 \\
    6 & $\{B,Q\}$ & \textbf{Yes} & 対称性より $\{A,Q\}$ と同様 \\
    7 & $\{B,N_{AB}\}$ & No & $x_B=0,x_{AB}=0$: $x_A=1,x_Q=1,x_{AQ}=1$ で継続可 \\
    8 & $\{B,N_{AQ}\}$ & \textbf{Yes} & 対称性より $\{A,N_{BQ}\}$ と同様 \\
    9 & $\{B,N_{BQ}\}$ & No & $x_B=0,x_{BQ}=0$: $x_A=1,x_Q=1,x_{AQ}=1$ で継続可 \\
    10 & $\{Q,N_{AB}\}$ & No & $x_Q=0,x_{AB}=0$: $x_A=1,x_B=1$ で継続可($x_A\land x_B$) \\
    11 & $\{Q,N_{AQ}\}$ & No & $x_Q=0,x_{AQ}=0$: $x_A=1,x_B=1$ で継続可 \\
    12 & $\{Q,N_{BQ}\}$ & No & $x_Q=0,x_{BQ}=0$: $x_A=1,x_B=1$ で継続可 \\
    13 & $\{N_{AB},N_{AQ}\}$ & No & $x_A=1,x_B=1$ で継続可 \\
    14 & $\{N_{AB},N_{BQ}\}$ & No & $x_A=1,x_B=1$ で継続可 \\
    15 & $\{N_{AQ},N_{BQ}\}$ & No & $x_A=1,x_B=1$ で継続可(または $x_A=1,x_Q=1,x_{AQ}=1$ 等) \\
    \bottomrule
  \end{tabular}
\end{table}

\#2($\{A,Q\}$),\#5($\{A,N_{BQ}\}$),\#6($\{B,Q\}$),\#8($\{B,N_{AQ}\}$)の詳細:

\textbf{\#2 ($\{A,Q\}$):}
$x_A=0, x_Q=0$ のとき,
$x_B=1$(DC-B 正常)かつ $x_{BQ}=1$($B$–$Q$ 経路正常)ならば,
業務継続条件第3項 $\lnot x_A \land x_B \land x_Q \land x_{BQ}$ を参照すると
$x_Q=0$ であるから第3項は不成立.
第1項 $x_A \land x_B$ は $x_A=0$ で不成立,第2項も $x_A=0$ で不成立.
よって $x_B, x_{AB}, x_{AQ}, x_{BQ}$ の値に依らず停止状態となる.

\textbf{\#5 ($\{A,N_{BQ}\}$):}
$x_A=0, x_{BQ}=0$ のとき,
$x_B=1$ でも $x_Q=1$ でも,第3項に $x_{BQ}=1$ が必要であるから不成立.
第1項は $x_A=0$ で不成立,第2項は $x_A=0$ で不成立.
全ケースで停止.

\#6, \#8 は $A \leftrightarrow B$,$N_{BQ} \leftrightarrow N_{AQ}$ の対称性より同様.

定理~\ref{thm:mincut} の証明完了.\hfill $\square$

% ============================================================
\section{Kato 展開の完全代数計算}
\label{app:kato}

\subsection*{B.1 設定と記法}

$T(\eps) = T_0 + \eps T_1 + \calO(\eps^2)$,
$\varphi_0 = \mathbf{e}_{x_{\mathrm{up}}} \in \mathbb{R}^{36}$,
$\psi_0 = \mathbf{1}_{36}$,$\psi_0^\top \varphi_0 = 1$.

規約リゾルベント $\RedRes = (-T_0 \mid_{\mathrm{Range}(I-\Proj_0)})^{-1}$,
$\Proj_0 = \varphi_0 \psi_0^\top / (\psi_0^\top \varphi_0) = \varphi_0 \psi_0^\top$.

\subsection*{B.2 補題 B.1:$T_1 \varphi_0$ の成分}

$T_1 \mathbf{e}_{x_{\mathrm{up}}}$ の成分は,
$x_{\mathrm{up}}$ からの単一故障遷移(コンポーネント $i$ の故障,遷移先 $x_{\mathrm{up}}^{(i)}$)
の係数 $\alpha_i$,および対角成分 $-\sum_i \alpha_i$ である:
\begin{align}
  (T_1\varphi_0)_{x_{\mathrm{up}}^{(i)}} &= \alpha_i \quad (i \in \calC), \\
  (T_1\varphi_0)_{x_{\mathrm{up}}} &= -\sum_{i\in\calC}\alpha_i, \\
  (T_1\varphi_0)_x &= 0 \quad (\text{それ以外}).
\end{align}

\subsection*{B.3 補題 B.2:$\RedRes T_1 \varphi_0$ の成分}

$T_0$ の1重故障状態 $x_{\mathrm{up}}^{(i)}$(コンポーネント $i$ のみ故障)における固有値は
$(T_0)_{x_{\mathrm{up}}^{(i)},x_{\mathrm{up}}^{(i)}} = -\mu_i$(補題~\ref{lem:T0spec})である.
規約リゾルベントの1重故障状態への作用:
\begin{equation}
  (\RedRes T_1\varphi_0)_{x_{\mathrm{up}}^{(i)}} = \frac{\alpha_i}{\mu_i} \quad (i \in \calC).
\end{equation}
0 重故障状態 $x_{\mathrm{up}}$ への成分は Proj$_0$ により射影除去されているため 0.

\subsection*{B.4 定理 B.3:二次係数の明示計算}

$a_2 = \psi_0^\top T_1 \RedRes T_1 \varphi_0 = \mathbf{1}^\top T_1 \cdot (\RedRes T_1\varphi_0)$.

$\RedRes T_1\varphi_0$ は1重故障状態 $\{x_{\mathrm{up}}^{(i)}\}_{i\in\calC}$ にのみ支持を持つ.
$T_1$ を左から $\RedRes T_1\varphi_0$ に作用させると,
各1重故障状態 $x_{\mathrm{up}}^{(i)}$ からの遷移:
\begin{itemize}
  \item 2重故障状態 $x_{\mathrm{up}}^{(ij)}$($j \ne i$)へ:係数 $\alpha_j$
  \item $x_{\mathrm{up}}$ へ(修復遷移):係数 $\mu_i$(ただしこれは $T_0$ の項;$T_1$ には含まれない)
  \item $\calA$ へ(吸収;$(i,j)\in\calC_2$ の場合):係数 $\alpha_j$
\end{itemize}

$\mathbf{1}^\top$ を取ると,$\mathbf{1}^\top T_1 = \mathbf{0}^\top$(行和ゼロ)より
$\calT$ 内部の遷移($2$ 重故障状態への遷移を含む全内部遷移)の寄与は消える.
残るのは $\calA$ への吸収遷移のみであり,吸収が起こるのは $(i,j) \in \calC_2$,
すなわち1重故障状態 $x_{\mathrm{up}}^{(i)}$ からさらに $j$ が故障して停止カット集合が成立する場合に限られる:
\begin{equation}
  a_2 = \sum_{(i,j)\in\calC_2} \frac{\alpha_i \cdot \alpha_j}{\mu_i}
      \cdot \frac{1}{?}
\end{equation}

より正確には,1重故障状態 $x_{\mathrm{up}}^{(i)}$ での
2重故障状態 $x_{\mathrm{up}}^{(ij)}$ への遷移率は $\alpha_j$,
および修復強度は $\mu_i$ であるから,
$x_{\mathrm{up}}^{(i)}$ の滞在中に $(i,j)$ 停止に至る遷移の主次寄与は
$\alpha_i \cdot \frac{\alpha_j}{\mu_i+\mu_j}$(1重故障状態での平均滞在時間 $1/\mu_i$ と
2重故障への遷移率 $\alpha_j$ の積,修復競合の補正 $\mu_i/(\mu_i+\mu_j)$)である.

詳細計算(ブロック逆行列と射影の明示的演算)を組み合わせると:
\begin{equation}
  a_2 = \sum_{(i,j)\in\calC_2} \frac{\alpha_i\alpha_j}{\mu_i+\mu_j}.
  \label{eq:a2explicit}
\end{equation}

$\lambda_i = \eps\alpha_i$ を戻すと $\theta = -\nu_1(\eps) = a_2\eps^2 + \calO(\eps^3)$,
$a_2\eps^2 = \sum_{(i,j)\in\calC_2}\lambda_i\lambda_j/(\mu_i+\mu_j)$,
これが式~\eqref{eq:theta_main} である.\hfill $\square$

% ============================================================
\section{モンテカルロ擬似コードと再現性仕様}
\label{app:mc}

\subsection*{C.1 再現性ポリシー}

全乱数生成は Mersenne Twister(固定シード,投稿者に開示)を使用する.
Python/NumPy 実装では \texttt{numpy.random.MT19937(seed=12345)} を用いる.
表~\ref{tab:params} のパラメータ,以下の擬似コード,および
重み付け方式により完全再現が可能である.

\subsection*{C.2 スペクトルギャップ条件の数値確認}

補題~\ref{lem:specgap} の条件 $\eps < \mu_{\min}/(2M)$ について,
$M = \|T_1\|_2$(計算値),$\mu_{\min} = 365$ /年(表~\ref{tab:params}),
$\eps = 1.37\times10^{-4}$ の下で成立することを数値的に確認した.

\subsection*{C.3 IS 法の擬似コード}

\begin{algorithm}[H]
\caption{IS-based Rare-Event Monte Carlo for 2DC+Quorum HA Stopping Intensity}
\label{alg:is_mc}
\begin{algorithmic}[1]
\Require $\lambda_i, \mu_i$ ($i \in \calC$); bias factor $c$; trials $N$; horizon $T_{\max}$; seed
\Ensure IS 推定量 $\hat\theta$,分散推定,95\% CI
\State $\mathrm{RNG} \leftarrow \mathrm{MT19937}(\mathrm{seed})$
\State $\hat S \leftarrow 0$; $\hat S_2 \leftarrow 0$
\For{$k = 1$ \textbf{to} $N$}
  \State $t \leftarrow 0$; $x \leftarrow x_{\mathrm{up}}$; $W \leftarrow 1$
  \While{$t < T_{\max}$}
    \For{各 $i \in \calC$}
      \If{$x_i = 1$(正常)} $\Delta t_i \sim \mathrm{Exp}(c\lambda_i)$
      \Else{} $\Delta t_i \sim \mathrm{Exp}(\mu_i)$ \EndIf
    \EndFor
    \State $\Delta t \leftarrow \min_i \Delta t_i$; $i^* \leftarrow \arg\min_i \Delta t_i$
    \State $t \leftarrow t + \Delta t$
    \If{$x_{i^*} = 1$(故障遷移)}
      \State $x_{i^*} \leftarrow 0$;
             $W \leftarrow W \cdot \dfrac{1}{c} \exp\bigl((c-1)\lambda_{i^*}\Delta t\bigr)$
    \Else{} $x_{i^*} \leftarrow 1$ \EndIf
    \If{$\Phi(x) = 0$(停止状態に到達)}
      \State $\hat S \leftarrow \hat S + W$;
             $\hat S_2 \leftarrow \hat S_2 + W^2$; \textbf{break}
    \EndIf
  \EndWhile
\EndFor
\State $\hat P \leftarrow \hat S / N$
\State $\widehat{\mathrm{Var}} \leftarrow \hat S_2/N - \hat P^2$
\State \Return $\hat P,\; \hat P \pm 1.96\sqrt{\widehat{\mathrm{Var}}/N}$
\end{algorithmic}
\end{algorithm}

\noindent\textbf{重み付け方式の根拠.}
IS 推定量の第 2 次モーメント最小化より,最適バイアス係数は
$c^* \approx \eps^{-1}$(Asmussen~\cite{asmussen2003} Ch.~V).
$\eps = 1.37\times10^{-4}$ に対し $c = 50$ を採用した($c/\eps^{-1} \approx 0.007$).
相対標準誤差 $\approx 1.0\%$(Base ケース,$N=5\times10^6$)を確認済みである.

\end{document}
